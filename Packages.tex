% Diese Datei enthält sämtliche zusätzlich eingebundenen Pakete

% URLs mit Zeilenumbruch ermöglichen
% \usepackage[hyphens]{url}

% PDF Dokumenteneigenschaften - Hyperref ermöglicht Meta-Informationen in der PDF
% https://www.namsu.de/Extra/pakete/Hyperref.html
% Links werden im Dokument schwarz dargestellt
\usepackage[pdftitle={DevOps und CD/CI},pdfsubject={Eine Uebersicht fuer Ansaetze aus der Praxis},pdfauthor={Danny Langenbach},pdfkeywords={Softwareentwicklung, Agile Modelle, Agiles Manifest, CI, CD, CI-CD-Pipeline,DevOps},pdfstartview=FitH,colorlinks=true,linkcolor=black,urlcolor=black,citecolor=black]{hyperref}

% https://tex.stackexchange.com/questions/3033/forcing-linebreaks-in-url
% https://tex.stackexchange.com/a/3034
% https://tex.stackexchange.com/a/10419
\PassOptionsToPackage{hyphens}{url}\usepackage{hyperref}

% Komplette PDFs einbinden
\usepackage{pdfpages}

% main=ngerman legt die Hauptsprache des Dokuments auf Neue Deutsche Rechtschreibung fest
% (Umlaute, Datum etc)
% englisch erlaubt es einzelne Abschnitte in Englisch zu definieren
\usepackage[main=ngerman, english]{babel}
\usepackage[T1]{fontenc}
\usepackage[utf8]{inputenc}  

% Für das Euro-Symbol
% \usepackage{eurosym}  \DeclareUnicodeCharacter{20AC}{\euro}

% Graphix ermöglicht die Einbindung von .jpeg,.jpg,.png usw.
\usepackage{graphicx}

% Kopzeile und Stil
\usepackage{fancyhdr}

\usepackage{caption}
% Fußnote
\usepackage{footmisc}

% Um das Abbildungsverzeichnis ins Inhaltsverzeichnis aufzunehmen
% Nottoc verhindert das sich das Inhaltsverzeichnis selbst als Inhalt nennt
% https://tex.stackexchange.com/a/297985
\usepackage[nottoc]{tocbibind}
% https://texblog.org/2013/04/29/latex-table-of-contents-list-of-figurestables-and-some-customizations/#totoc

% Package für Gesamtseitenzahl
\usepackage{lastpage}

% tabularx für Tabellen
\usepackage{tabularx}

% Für Blindtext
\usepackage{lipsum}

% Zitate APA Style - 6. Version
% Muss auskommentiert werden, wenn IEEEtran verwendet werden soll
% \usepackage{apacite}

% Für das Abkürzungsverzeichnis
% Ermöglicht die Verwendung von \acs im Text und gibt nur verwendete Abkürzungen im Glossar aus
% \usepackage[printonlyused,withpage]{acronym}
% printonlyused = Nur verwendete Acronyme angeben
% WithPage = Seite der Abkürzung im Dokument

% nopostdot = Kein Punkt nach der Abkürzung
% nonumberlist = Keine Seitenzahlen ausgeben, wo die Abkürzungen verwendet werden.
% toc = true nimmt Eintrag ins Inhaltsverzeichnis auf, ohne Notwendigkeit von \addcontentsline
% siehe https://tex.stackexchange.com/a/156742
% \usepackage[acronym,toc=true]{glossaries}
\usepackage[acronym,toc=true]{glossaries}

% \usepackage{xparse}

% Für mehrzeilige Kommentare
\usepackage{verbatim}

% Für Textoverlays wie z.B. "Entwurf" oder "Vertraulich"
% Gestaltung des Wasserzeichens unter /Layout/Wasserzeichen
% \usepackage[firstpage]{draftwatermark} - Wasserzeichen nur auf der ersten Seite
% \usepackage[nostamp]{draftwatermark} - Kein Wasserzeichen irgendwo im Dokument
% \usepackage{draftwatermark}

% Für durchgestrichenen Text
\usepackage{ulem}