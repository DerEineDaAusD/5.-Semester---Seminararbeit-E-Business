\subsection{Generische Softwareentwicklunsmodelle}
% \addcontentsline{toc}{subsection}{Generische Softwareentwicklunsmodelle}
\label{Generische Softwareentwicklunsmodelle}
Grundsätzlich folgt jeder Prozess der Softwareentwicklung den gleichen grundlegenden Schritten. Diese werden in der nachstehenden Grafik in Anlehnung an \cite[Abb. 9.1]{Ludwig_Lichter_2013}: 

Jede der Phasen erfüllt dabei feste Aufgaben:
\begin{table}
	Phase & Aufgabe\\
	Analyse & Aufnahme der Anforderungen an eine Software und der (Projekt-) Umgebung. \glqq{}Konkretisierung der Analyse\grqq{} \cite[S. 155]{Ludwig_Lichter_2013}\\
	Spezifikation & Präzisierung / Verschriftlichung der Analyse\\
	Grobentwurf & Festlegung der groben Programmstruktur\\
	Feinentwurf & Präzisierung des Grobentwurfs, Festlegung der genauen Implementierung\\
	Codierung & Umsetzung des Feinentwurfs in Code, Tests des Codes (Unit-Codes)\\
	Integration, Test und Abnahme & Einpassung des Produktes in die Produktivumgebung\\
\end{table}
Diese Phasen sind in sich abgeschlossen und haben definierte Übergabepunkte (Artefakte) bei einem Wechsel zwischen zwei Phasen \cite[S. 155]{Ludwig_Lichter_2013}. Hierzu sei erwähnt, dass es sich um eine vereinfachende und idealisierte Betrachtung handelt. Der feste \textit{Top-to-Bottom}-Ansatz ist für die meisten praktischen Anwendungsfälle zu simpel und wird veränderbaren oder wechselnden Anforderungen nicht gerecht. Diese Einschränkung hat bereits  Dr. Winston W. Royce als Erfinder des Wasserfallmodells \cite{royce1987managing} erkannt und ihr Rechnung getragen. Royce selbst hat eine Wiederholung der Modelldurchläufe vorgeschlagen \cite{Larmann_Basili_2003} und damit das lineare Modell iterativ erweitert. Die nachstehende Grafik verdeutlich dies in Anlehnung an \cite[Abb. 3]{royce1987managing}:



Ein iterativer Ansatz ermöglicht es, auf Veränderungen interner und externer Faktoren in einem Softwareprojekt zu reagieren. Dabei ist es zunächst unerheblich, welche Faktoren dies sind (Externe wie z.B. veränderte Anforderungen oder interne wie z.B. Wechsel einer Programmierungsmethode), entscheidend ist nur die Phase, in welcher die Veränderungen auftreten. An diesem Punkt kann die einzelne Phase (oder alle bis dahin durchlaufenen Phasen) erneut durchlaufen werden.
Dieser lineare Ablauf wird im Spiralmodell weiter erweitert:

// Originalquelle Spiralmodell
\cite[Abb. 2]{boehm_spiral_1988} 

Von diesen Modellen können weitere Formen abgeleitet werden. Daher werden sie als generische Entwicklungsmodelle bezeichnet.
