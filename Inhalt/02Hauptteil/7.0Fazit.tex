\section{Fazit}
% \addcontentsline{toc}{section}{Fazit}
\label{Fazit}
Das Thema \gls{DevOps} ist aktuell. Die Anzahl wissenschaftlicher Publikationen über \gls{DevOps} steigt seit einigen Jahren permanent \cite{leite_survey_2020} und durch die Präsenz von Lösungen mit niedrigen Einstiegshürden (z.B. Nodered \cite{nodered_about}, Github \cite{DevOps_Definition_Microsoft}/Gitlab \cite{gitlab_devops}) werden die Arbeits- und Funktionsweisen für einen breiteren Personenkreis zugängig. Dabei muss es sich nicht mehr nur um Programmierer handeln, wie das Beispiel von NodeRed zeigt.
\gls{DevOps} ist außerdem Wettberwerbs- und Wirtschftsfaktor: Entwicklung mit \acrshort{CI}-Techniken führen zu schnelleren Release-Zyklen und höheren Release-Zahlen \cite{forsgren_devops_2015} \cite{hilton_usage_2016}. Damit ermöglicht \gls{DevOps} einerseits eine höhere Innovationsdichte und schnellere Reaktionen auf neue Trends (Vgl. Verkürzung der \gls{Time-to-Market} in Abs. \ref{Betriebswirtschaftliche Potenziale}), andererseits aber auch geringere Reaktionszeiten im Fall von Fehlern oder Problemen \cite{zhao_impact_2017}. \gls{DevOps} und die dadurch im Entwicklungsprozess eingebrachte Transparenz ist damit Marketingvorteil (\acrshort{USP}).
\gls{DevOps} ist außerdem integrativ und kollaborativ, sowie interdisziplinär. Über den gesamten Weg einer \acrshort{CI}/\acrshort{CD}-Pipeline berührt das Thema unterschiedlichste Unternehmensbereiche von Management über Entwicklung und Administration bis zum Personalwesen. Gleichzeitig umfasst das Thema damit Gebiete von Management- und Mitarbeiterqualifikation und notwendigen Softskills bis hin zu Unternehmensstrukturen, Ablauf- und Aufbauorganisation. Zu diesen Managementherausforderungen kommen dann die technischen Aspekte wie der Aufbau von Entwicklungs- und Testumgebungen, Releasezyklen und -Managment und die Administration der \acrshort{CI}/\acrshort{CD}-Pipeline im Wirkbetrieb \cite[Abb. 5 bis 8]{leite_survey_2020}. 
Mit diesen umfangreichen Implikationen in die unterschiedlichsten Aspekte eines Unternehmens ist  \gls{DevOps} vielmehr eine Unternehmenskultur statt eine Ansammlung einzelner Prozesse und Praktiken \cite{DevOps_Definition_Microsoft} \cite{DevOps_Definition_AWS}. Unternehmenskultur impliziert damit auch, dass diese Philosphie und alle damit verbundenen Bereich im Unternehmen vom Management und Führungsebene angefangen gelebt werden müssen. Damit geht es für das Management vor allem darum, eine Umgebung zu schaffen, in der die \gls{DevOps}-Prinzipien von Kollaboration, Transparenz, Fehlertoleranz bis zu interdisziplinärem Denken gelebt werden können \cite[Abschnitt 7.2]{leite_survey_2020}.
Aus diesem Punkt ergeben sich dann auch die Herausforderungen bei der Etablierung einer \acrshort{DevOps}-Kultur. Da eine Kultur etwas Immaterielles darstellt, gibt es keine Patentlösungen, wie derartige Umgebungen in der Praxis umgesetzt werden soll.
Entsprechend ist der Weg bis zur vollständigen Etablierung von \gls{DevOps} geprägt von \glqq{}trial and error\grqq{}. Da es nicht \textit{die Musterlösung} gibt, muss jedes Unternehmen selbst herausfinden, welche Strategien, Methoden und Werkzeuge bei der Etablierung von \gls{DevOps} geeignet sind. Auf Grundlage der zuvor genannten Quellen kann jedoch festgehalten werden, dass allein die Verwendung einer Versionsverwaltung noch kein \acrshort{CI} darstellt, sondern dieser Ansatz wesentlich weiter gefasst ist. Dies birgt natürlich auch das Risiko des Scheiterns und des damit einhergehenden potenziellen Verlusts bereits getätigter Investitionen, entweder in Personal oder Infrastruktur. Zusätzlich lässt sich argumentieren, dass \gls{DevOps} und \acrshort{CI}/\acrshort{CD} nicht für den Einsatz in jedem Projekt und jeder Umgebung geeignet ist, doch die dieser Arbeit zu Grunde liegenden Quellen bestätigen dies nicht \cite[Abb. 2 und 3]{hilton_usage_2016}. Zusätzlich können Investitionsrisiken minimiert, bzw. ausgelagert werden, sofern auf fertige Plattformlösungen Dritter (Vgl. \ref{UseCases mit proprietärer Software}) zurückgegriffen wird.\newline
Zusammenfassend kann abschließend festgehalten werden, dass in der \gls{DevOps}-Philosophie und dem zugrunde liegenden agilen Manifest \cite{beck_manifest_2001} große Potenziale für Softwareentwicklung in jedem Maßstab liegen. Durch die hohe Aktualität des Themas sind Lösungen mit niedrigen Einstiegskosten weit verbreitet und einfach zugängig. Damit lassen sich Investitionsrisiken im Vergleich zu den Potenzialen minimieren.Dennoch ist nicht ausgeschlossen, dass die Etablierung einer \acrshort{DevOps}-Kultur in einem Unternehmen auch scheitern kann. Dies ist ein Aspekt der im Umfang dieser Arbeit nicht tiefergehend bearbeitet wurde.
Die Frage, wie \gls{DevOps} daher in Unternehmen eingeführt werden kann und welche Schritte dafür als technisch-organisatorische Maßnahmen zur Risikovermeidung nötig sind, ist damit eine relevante Frage für weitere Forschung. \gls{DevOps} insgesamt bietet ein gewaltiges Feld für weitere Forschung sowohl qualitativer (Fallstudien, Experimente) und quantitativer Art (Literaturanlyse) und mit der weitergehenden Verbreitung dieser Prinzipien ist nicht absehbar, das dieser Trend alsbald endet.