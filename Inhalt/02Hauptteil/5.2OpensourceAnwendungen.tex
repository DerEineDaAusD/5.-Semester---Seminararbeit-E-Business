\subsection{Anwendungen mit Opensource-Software}
% \addcontentsline{toc}{subsection}{Anwendungen mit Opensource-Software}
\label{Anwendungen mit Opensource-Software}
Im Feld der Opensource-Lösungen gibt es, wie in vielen Fällen, eine große Anzahl breit gestaffelter Lösungen für unterschiedlichste Anwendungen und Spezialisierungen. Dieser Abschnitt betrachtet daher wie zuvor wieder nur einige der prominentesten Lösungen als Beispiele.
Als Ableger der quelloffenen Versionsverwaltung Git, ist Gitlab ein großer Vertreter der \acrshort{CI}/\acrshort{CD}-Tools im Opensource-Bereich. Wie zuvor schon bei den proprietären bietet auch GitLab die unterschiedlichsten Tools zum Aufbau einer kompletten \gls{DevOps}-Umgebung. Im Gegensatz zu den Lösungen von Microsoft und Amazon handelt es sich hier jedoch um verschiedene Funktionen einer Anwendung, nicht um unterschiedliche Produkte auf/in einer Plattform \cite{gitlab_devops}. Gitlab selbst bietet neben einer Opensource-Version auch kostenpflichtige Inhalte an \cite{gitlab_pricing}. Aufgrund seiner Herkunft aus dem Opensource-Bereich ist es in dieser Arbeit jedoch bei den Opensource-Lösungen geführt.
Eine weitere, ebenfalls in der Opensource-Community weit verbreitete Lösung ist Jenkins. Im Gegensatz zu Gitlab handelt es sich bei Jenkins nicht um eine integrierende Plattformlösung, sondern um einen Buildserver, der über unterschiedlichste Plugins und Skript-Schnittstellen zum Aufbau einer Pipeline genutzt werden kann \cite{jenkins_about}.
Abschließend ist noch festzuhalten, dass viele der hier vorgestellten Lösungen untereinander und zueinander interoperabel sind und über verschiedenste Schnittstellen miteinander verwendet werden können. Das die vorgestellten Lösungen in Konkurrenz zueinander stehen, schließt eine gemeinsame Verwendung nicht aus.