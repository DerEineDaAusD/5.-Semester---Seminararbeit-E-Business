\subsection{Betriebswirtschaftliche Potenziale}
% \addcontentsline{toc}{subsection}{Betriebswirtschaftliche Potenziale}
\label{Betriebswirtschaftliche Potenziale}
Wie zuvor angedeutet, sind agile Softwareentwicklungsmodelle darauf ausgelegt, lauffähige Software in kurzer (Vgl. Sprint-Zyklen bei \gls{SCRUM}) Zeit auszuliefern und dabei permanent ändernde Gegebenheiten aufzunehmen (Vgl. \ref{Agile Grundsaetze - Das agile Manifest}). Tatsächlich ist dieser Gedanke in den Prinzipien des agilen Manifests soweit etabliert, dass \glqq Auftraggeber, Entwickler und Benutzer [...] ein gleichmäßiges [Entwicklungs-]Tempo auf unbegrenzte Zeit halten können\grqq{}\cite{beck_prinzipien_2001}.\newline
\gls{DevOps} verkürzt vor diesem Hintergrund aggressiv die \gls{Time-to-Market} und sorgt für kürzere Innovations-, Entwicklungs- und Release-Zyklen \cite{forsgren_devops_2015}. Durch die Automatisierung in Teilen (oder der ganzen) \acrshort{CI}/\acrshort{CD}-Pipeline entfallen Schritte mit hohem manuellen Arbeitsaufwand, wie Tests, Bereitstellung von Testumgebungen und Vorbereitung(en) für den Release. Dies verkürzt nicht nur in einem Schritt die erwähnte \gls{Time-to-Market}, sondern senkt mit steigendem Automatisierungsgrad auch die Bereitstellungskosten je Softwarerelease \cite{forsgren_devops_2015}.
Wird ein Teil der dafür benötigen Infrastruktur zusätzlich noch ausgelagert (z.B. in eine Cloudplattform), können die eigenen Betriebskosten noch weiter gesenkt werden. Dieser Asepkt wird in Abschnitt \ref{UseCases mit proprietärer Software} noch näher betrachtet.\newline
Mit kürzeren Releasezyklen geht auch eine höhere Innovationsfähigkeit und bessere Desaster-Recovery im Fehlerfall einher. Einerseits können neue Entwicklungen und Trends kurzfristig aufgegriffen werden, andererseits kann durch den beschleunigten Release im Fehlerfall schneller reagiert werden \cite{forsgren_devops_2015}.
Gleichzeitig bietet der abteilungsübergreifende, interoperable Ansatz von \gls{DevOps} große Potenziale für Lernprozesse und Transparenz sowohl zwischen im Unternehmen beteiligten Abteilungen als auch zwischen einem Unternehmen und seinen Kunden.
Auf Basis der \acrshort{CI}/\acrshort{CD}-Pipeline kann zu jedem Zeitpunkt Transparenz über die einzelnen Entwicklungsschritte hergestellt werden und diese Transparenz in Kombination mit der kontinuierlichen Verfügbarmachung der Software kann ein Alleinstellungsmerkmal (Unique Selling Proposition, \acrshort{USP}) gegenüber dem Endkunden sein. Dies bietet auch auch Potenziale für Kunden mit hohen Anforderungen an formale Strukturen, Zertifizierungen etc. \cite[Kap. 7.2]{leite_survey_2020}.