\subsection{Methode}
% \addcontentsline{toc}{subsection}{Methode}
\label{Methode}
Im Rahmen dieser Seminararbeit soll über eine qualitative Literaturrecherche dargestellt werden, welche genauen Werkzeuge und Methoden sich hinter den in der Einleitung genannten Schlagworten verbergen und welche Vorteile und Potenziale diese bieten.
Für einen theoretischen Rahmen werden dabei zuerst die Modelle und Ideen der klassischen Softwareentwicklung sowie der agilen Softwareentwicklung dargestellt. Auf dieser Grundlage erfolgt eine Definition der in der Einleitung genannten Schlagwort von \gls{DevOps} und \acrshort{CD}/\acrshort{CI}.
Anschließend werden sowohl die technischen als auch betriebswirtschaftlichen Potenziale der genannten Konzepte betrachtet. Darauffolgend wird die Umsetzung in der Praxis anhand von Fallbeispielen (Usecases) erläutert. Teil dieser Erläuterung ist die exemplarische Darstellung, welche konkreten Lösungen und Produkte aus dem kommerziellen oder Opensource-Bereich  auf dem Markt vorhanden sind.
Abschließend wird evaluiert welche Voraussetzungen aus der Praxis  für \acrshort{CD}/\acrshort{CI}-Pipelines definiert werden können und es erfolgt ein Fazit.