\subsection{DevOps und CI/CD}
% \addcontentsline{toc}{subsection}{DevOps und CI/CD}
\label{DevOps und CI/CD}
\gls{DevOps} ist der Ansatz der agilen Softwareentwicklung weitergedacht in den Betriebsalltag einer Software. \gls{DevOps} selbst ist ein Kunstwort aus \glqq{}Development\grqq{} und \glqq{}Operations\grqq{}. \gls{DevOps} bezeichnet die direkte Integration von Unternehmenskultur und unterstützender \acrshort{IT}-Prozesse in den Softwareentwicklungsprozess \cite{halstenberg_devops_2020}. Diese unterstützenden \acrshort{IT}-Prozesse umfassen oftmals Aufgaben wie Bereitstellung der Umgebung für den Test- und Wirkbetrieb einer Software, Verfügbarmachung einer Software für den Endkunden und Betreuung der Anwendung im Alltag oder auch Qualitätssicherung \cite{DevOps_Definition_AWS}. \acrshort{DevOps} soll dabei die Entwicklung und Bereitstellung von Software erleichtern und die damit verbundene Time-to-Market (Bereitstellungszeit) erheblich verkürzen. Gleichzeitig dienen diese Vorteile auch der Unterstützung anderer Phasen im Softwarelebenszyklus, insbesondere dem Patch-Management und der Wartung \cite{DevOps_Definition_Microsoft}.\newline
\gls{DevOps} ist dabei mehr als nur ein Neuordnen vorhandener Abläufe. Der Ansatz umfasst nicht nur technische Änderungen, sondern Änderungen der gesamten Unternehmenskultur: Vorher getrennte Abläufe werden miteinander vernetzt und vorher getrennte Zuständigkeiten entfallen. Abteilungen und Organisationseinheiten müssen nicht nur mehr miteinander kommunizieren, sondern müssen füreinander transparent sein. Zudem erhalten sie untereinander Einblick in Prozesse und Gebiete, für die sie eigentlich nicht zuständig sind \cite{DevOps_Definition_Microsoft} \cite{DevOps_Definition_AWS}.\newline
Vor diesem Hintergrund sind die Schlagworte \glqq{}Continuous Development\grqq{} und \glqq{}Continuous Integration\grqq{} Ausprägungen der \gls{DevOps}-Philosophie. Continuous Development (\acrshort{CD}) bezeichnet den iterativ-inkrementellen Ansatz der Softwareentwicklung, der allen agilen Modellen und damit auch dem \gls{DevOps}-Ansatz zugrunde liegt. Dabei geht es um die permanente und zyklische Weiterentwicklung der Code-Basis einer Software. Continuous Integration (\acrshort{CI}) bezeichnet die dabei stattfinde, fortlaufende Integration von Änderungen und Erweiterungen in die bestehende Anwendungen. \acrshort{CD} verfügt aber noch über eine weitere Bedeutung: Continuous Delivery. Delivery, also die Auslieferung oder Bereitstellung der Software, kann sich dabei auf die Bereitstellung des fertigen Produktes z.B. in Form eines Datenträgers oder Installationsimages beschränken. Der Ansatz kann aber auch erheblich weitergedacht werden. Dadurch sind Szenarien möglich, bei denen eine Änderung in einer direkten Kette vom ausführenden Programmierer über die Versionsverwaltung, Build-Umgebung, Test-Umgebung und Bereitstellung direkt in ein Produktivsystem im eigenen Unternehmen oder bei einem Kunden geladen wird. Diese Kette wird als \acrshort{CI}/\acrshort{CD}-Pipeline bezeichnet \cite{DevOps_Definition_Microsoft}  \cite{DevOps_Definition_AWS} \cite{atlassian_CivsCDvsCD_nodate} \cite{NodeRed_CICD_nodate}. 