\section{Potenziale von CI/CD}
% \addcontentsline{toc}{section}{Potenziale von CI/CD}
\label{Potenziale von CI/CD}
Der nachfolgende Abschnitt betrachtet die Potenziale, die der Einsatz von \acrshort{CI}/\acrshort{CD} sowohl aus betriebswirtschaftlicher als auch aus technischer Sicht bietet.