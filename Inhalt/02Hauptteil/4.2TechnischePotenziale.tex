\subsection{Technische Potenziale}
% \addcontentsline{toc}{subsection}{Technische Potenziale}
\label{Technische Potenziale}
Ebenso wie betriebswirtschaftlich, bietet eine \gls{DevOps}-Kultur auch technische Vorteile. Dies umfasst auch Erleichterungen für die technischen Mitarbeiter, wie Programmierer oder Administratoren. \gls{DevOps} hat einen stark Mensch-bezogenen Ansatz in seinen Grundsätzen des offenen und gleichberechtigen Zusammenarbeitens. Integraler Bestandteil von \gls{DevOps} ist das Auflösen von \glqq{}Silos\grqq{} \cite{leite_survey_2020}, also Bastionen begrenzten Wissens. Dieser kollaborative Ansatz bietet für Mitarbeiter einerseits die Möglichkeit, neue Erfahrungen zu sammeln und Interessen an Themen jenseits des eigenen Fachbereiches auszuleben \cite{leite_survey_2020}. Da \gls{DevOps} zeitgleich auf die Grundgedanken des agilen Manifests zurückgeht, ist die Idee regelmäßiger Arbeitszeiten und der Vermeidung exzessiver Belastung an einzelnen Stellen immanent (Vgl. \glqq{}gleichmäßiges Tempo\grqq{}, \ref{Betriebswirtschaftliche Potenziale}).
Zeitgleich entlastet \acrshort{CI}/\acrshort{CD} über den Grad der Automatisierung und die Art der Architektur die einzelnen Entwickler: Software muss so gebaut sein, dass sie in Teilen iterativ verändert werden kann, ohne Auswirkungen auf das Gesamtprojekt zu haben (\glqq{}Microservice\grqq{} \cite[S. 14]{leite_survey_2020}). Die dadurch in die Software eingebrachte Modularität macht es nicht mehr erforderlich, jede Facette eines Projektes zu kennen. Änderungen am Code werden dadurch einfacher und die Einstiegshürden für neue oder fachfremde Entwickler geringer. Dies betrifft ebenfalls den notwendigen Aufwand zur Fehlersuche und -Beseitigung (Bugfixing) \cite{hilton_usage_2016}.

//Todo
% zhao_impact_2017
% rahman_characterizing_2018
