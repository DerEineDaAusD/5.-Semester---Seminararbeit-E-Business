\subsection{Agile Grundsätze - Das agile Manifest}
% \addcontentsline{toc}{subsection}{Agile Grundsätze - Das agile Manifest}
\label{Agile Grundsätze - Das agile Manifest}
Die zuvor aufgezeigten Defizite der klassischen Softwareentwicklung führten im Jahr 2001 zur Entstehung eines neuen Entwicklungsansatzes. Unter dem Schlagwort des agilen Manifests wurden neue Ideen für eine flexiblere und effektivere Art der Softwareentwicklung veröffentlicht. Damit sollte dem häufigen Scheitern von Softwareprojekten an den Gegebenheiten der Realität (wechselnde Umgebungsbedingungen und Anforderungen während der Projektlaufzeit, schwierige bzw. unplanbare Faktoren) Rechnung getragen werden (Vgl. \glqq{}Software-Krise\grqq{} und CHAOS-Report \cite{noauthor_standish_1995}). Zeitgleich war es auch ein Anliegen des agilen Manifests, menschliche Interaktion über formale Organisation zu stellen \cite{beck_manifest_2001}.
Das agile Manifest definiert die folgenden Grundsätze:
\begin{itemize}
    \item Individuen und Interaktionen über Prozesse und Werkzeuge
    \item Funktionierende Software über umfassende Dokumentation
    \item Zusammenarbeit mit dem Kunden über Vertragsverhandlung
    \item Reagieren auf Veränderung über das Befolgen eines Plans
\end{itemize}
Bei diesen Grundsätzen ist auch ganz bewusst der Gedanke aufgenommen, das die formulierte Aspekte höher eingeschätzt werden, als die zuvor etablierten Punkte. Diese werden jedoch nicht übergangen, sondern müssen weiterhin (mit entsprechender Priorität) berücksichtigt werden.
Dieser Ansatz sollte eine Lösung für die diversen Probleme der monolithischen Entwicklungsmodelle sein und definiert Softwareentwicklung als einen permanenten Prozess, der nicht mit der Abnahme eines Projektes beendet ist.