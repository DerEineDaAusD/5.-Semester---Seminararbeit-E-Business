\section{Benötigte Funktionen für eine CI/CD-Pipeline}
% \addcontentsline{toc}{section}{Benötigte Funktionen für eine CI/CD-Pipeline}
\label{Benoetigte Funktionen für eine CI/CD-Pipeline}
Die vorangegangenen Abschnitte haben einen Eindruck vermittelt, wie langwierig der Weg von klassischer über agile Softwareentwicklung hin zu \gls{DevOps} und \acrshort{CI}/\acrshort{CD} war. Die Abschnitte \ref{UseCases mit proprietärer Software}  und \ref{UseCases mit Opensource-Software} haben einen Eindruck vermittelt, welche Vielzahl von Anwendungen es in diesem weiten Themenfeld gibt. Abschließend ist nun zu betrachten, welche Elemente zwischen den Lösungen gleich und damit integraler Bestandteil einer \acrshort{CI}/\acrshort{CD}-Pipeline sind.\newline