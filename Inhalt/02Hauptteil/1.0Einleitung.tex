\section{Einleitung}
% \addcontentsline{toc}{section}{Einleitung}
\label{Einleitung}
20 Jahre nach der Veröffentlichung des agilen Manifests \cite{eckstein_20_2021} hat agile Softwareentwicklung nicht nur den Bereich der Softwareentwicklung selbst sondern auch noch viele Bereiche darüber hinaus grundlegend verändert, beeinflusst und weiterentwickelt. Zusätzlich zur Entwicklung von iterativen und agilen Modellen (allen voran das Flagschiff \gls{SCRUM} der Softwareentwicklung im Gegensatz zu den zuvor verwendeten linear-monolithischen Modellen, hat der agile Ansatz auch Bereiche wie die IT-Administration und teilweise auch Unternehmens- und Geschäftsmodelle beeinflusst.
Software und Softwareentwicklung wurde von einem (oder mehreren) monolithischen Projekt(en) zu einem fortlaufenden Prozess, dessen Entwicklung nie vollständig zu Ende ist. Dieser Ansatz hat spätestens mit der Idee von \grqq{}Windows as a Service\glqq{} \cite{jaimeo_kurzanleitung_2021} auch einer breiten Öffentlichkeit bewusst.\newline
Agile Methoden der Softwareentwicklung bedingen neue Entwicklungsstile und Arbeitsansätze. Software fortlaufend weiterzuentwickeln  macht die Fähigkeit zur permanenten Erweiterbarkeit zu einer Voraussetzung, bekannt unter den Schlagwörtern \glqq{} Continuous Development\grqq{} und \glqq{} Continuous Integration\grqq{}  (\acrshort{CD} / \acrshort{CI}). Geschäftsmodelle wie der zuvor erwähnte \glqq{}Windows as a Service\grqq{}-Ansatz oder auch generell die verschiedenen cloudbasierten Hostingansätze (Vgl. \acrshort{PaaS} und \acrshort{SaaS}) bedingen zugleich noch einen weiteren Aspekt als hinreichende Voraussetzung für das Funktionieren eines derartigen Geschäftsmodells: Kontinuierliche Verfügbarmachung - \glqq{}Continuous Delivery\grqq{} (\acrshort{CD}).\newline
Im Kontext dieser Herausforderungen wird häufig das Konzept \glqq{}\acrshort{DevOps}\grqq{} geannt und von einer \acrshort{CI}/\acrshort{CD}-Pipeline (teilw. auch \acrshort{CI}/\acrshort{CD}/\acrshort{CD}).
Doch was diese Schlagworte genau bedeuten oder wie überhaupt ihre Umsetzung in der Praxis aussieht, bleibt
häufig vage. Trotzdem hat sich auf dem Markt eine Vielzahl entsprechender Tools etabliert.
Diese Vielfalt reicht von den großen kommerziellen Anbietern wie Microsoft (Azure) und Amazon
(AWS) bis in den OpenSource-Bereich (z.B. Jenkins) und von riesigen Plattformen (z.B. RedHat Enterprise) bis hin zu Anwendungen, mit denen schon einzelne Personen ganze \acrshort{CI}/\acrshort{CD}-Pipelines administrieren können (z.B. Gitlab) oder auch Personen ohne Programmierkenntnisse einen niederschwelligen Einstieg finden (z.B. NodeRed).