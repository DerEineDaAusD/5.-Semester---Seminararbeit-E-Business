\section{Einleitung}
% \addcontentsline{toc}{section}{Einleitung}
\label{Einleitung}
20 Jahre nach der Veröffentlichung des agilen Manifests \cite{eckstein_20_2021} hat agile Softwareentwicklung nicht nur den Bereich der Softwareentwicklung selbst sondern auch noch viele Bereiche darüber hinaus grundlegend verändert, beeinflusst und weiterentwickelt. Über den Bereich der Softwareentwicklung hinaus, hat der agile Ansatz auch Bereiche wie die IT-Administration und teilweise auch Unternehmens- und Geschäftsmodelle beeinflusst. Hier ist insbesondere das \gls{SCRUM}-Modell als Flaggschiff zu nennen.
Software und Softwareentwicklung wurde von einem (oder mehreren) monolithischen Projekt(en) zu einem fortlaufenden Prozess, dessen Entwicklung nie vollständig zu Ende ist. Dieser Ansatz ist allerspätestens mit der Idee von \grqq{}Windows as a Service\glqq{} \cite{jaimeo_kurzanleitung_2021} auch einer breiten Öffentlichkeit bekannt geworden.\newline
Agile Methoden der Softwareentwicklung bedingen neue Entwicklungsstile und Arbeitsansätze. Software fortlaufend weiterzuentwickeln  macht die Fähigkeit zur permanenten Erweiterbarkeit zu einer Voraussetzung, bekannt unter den Schlagwörtern \glqq{} Continuous Development\grqq{} und \glqq{} Continuous Integration\grqq{}  (\acrshort{CD} / \acrshort{CI}). Geschäftsmodelle, wie der zuvor erwähnte \glqq{}Windows as a Service\grqq{}-Ansatz oder auch generell die verschiedenen cloudbasierten Hostingansätze (Vgl. \acrshort{PaaS} und \acrshort{SaaS}) bedingen zugleich noch einen weiteren Aspekt als hinreichende Voraussetzung für das Funktionieren eines derartigen Geschäftsmodells: Kontinuierliche Verfügbarmachung - \glqq{}Continuous Delivery\grqq{} (\acrshort{CD}).\newline
Im Kontext dieser Herausforderungen wird häufig das Konzept \glqq{}\gls{DevOps}\grqq{} genannt. Teil dieses Konzeptes sind die Begriff \acrshort{CI}/\acrshort{CD} (teilw. auch \acrshort{CI}/\acrshort{CD}/\acrshort{CD}) und \acrshort{CI}/\acrshort{CD}-Pipeline.
Doch was diese Schlagworte genau bedeuten oder wie überhaupt ihre Umsetzung in der Praxis aussieht, bleibt häufig vage. Trotzdem hat sich auf dem Markt eine Vielzahl entsprechender Tools etabliert.
Diese Vielfalt reicht von den großen kommerziellen Anbietern wie Microsoft (Azure) und Amazon (AWS) bis in den OpenSource-Bereich (z.B. Jenkins) und von riesigen Plattformen (z.B. RedHat Enterprise) bis hin zu Anwendungen, mit denen schon einzelne Personen ganze \acrshort{CI}/\acrshort{CD}-Pipelines administrieren können (z.B. Gitlab) und auch Personen ohne Programmierkenntnisse einen niederschwelligen Einstieg finden (z.B. NodeRed).