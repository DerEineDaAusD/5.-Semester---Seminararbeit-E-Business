\label{sec:Abkuerzungsverzeichnis}
% Beide Befehle mit Option toc = true beim Paket in der Packages.tex erledigt.
% \section*{Abkürzungsverzeichnis und Glossar}
% \addcontentsline{toc}{section}{Abkürzungsverzeichnis}

% Abschnittspezifische Header mit Abschnittsangabe oben rechts
\thispagestyle{Abkuerzungsverzeichnis}

% Abkürzungen definieren
\newacronym{EDV}{EDV}{Elektronische Datenverarbeitung}
\newacronym{POC}{POC}{Proof Of Concept}
\newacronym{HA}{HA}{High Aviability / Hochverfügbarkeit}
\newacronym{IaaS}{IaaS}{Infrastructure as a service}
\newacronym{SaaS}{SaaS}{Software as a Service}
\newacronym{OS}{OS}{Operating Systems / Betriebssystem}
\newacronym{VMM}{VMM}{Virtual Machine Manager}
\newacronym{VM}{VM}{Virtual Machine / Virtuelle Maschine}
\newacronym{CE}{CE}{Containerengine / Container-Engine}
\newacronym{CPU}{CPU}{Central Processing Unit / Zentrale Recheneinheit (eines Computers)}
\newacronym{RAM}{RAM}{Random Access Memory / Wahlfreier Zugriffsspeicher}
\newacronym{IoT}{IoT}{Internet of Things / Internet der Dinge}
\newacronym{UFS}{UFS}{Union File System}
\newacronym{AWS}{AWS}{Amazon Web Services}
\newacronym{ITU}{ITU}{International Telecommunication Union}
\newacronym{SIP}{SIP}{Session Initiation Protocol}
\newacronym{UAC}{UAC}{User Account Control / Nutzerkontenkontrolle}
\newacronym{CA}{CA}{Certificate Authority / Zertifizierungstelle}
\newacronym{STUN}{STUN}{Session Traversal Utilities for NAT}
\newacronym{TURN}{TURN}{Traversal Using Relays around NAT}
\newacronym{NAT}{NAT}{Network Address Translation}
\newacronym{IETF}{IETF}{Internet Engineering Task Force, eine offene Gesellschaft zur Schaffung von Standards im Internet. Vgl. \url{https://ietf.org/about/}}
\newacronym{ISDN}{ISDN}{Integrated Systems Digital Network / Dienstintegrierendes Digitales Netzwerk}
\newacronym{VoIP}{VoIP}{Voice-over-IP}
\newacronym{BSI}{BSI}{Bundesamt für Sicherheit in der Informationstechnik}
\newacronym{VAF}{VAF}{Bundesverband Telekommunikation e.V.}
\newacronym{QoS}{QoS}{Quality of Service}
\newacronym{IP}{IP}{Internet Protocol}
\newacronym{Payload}{Payload}{Nutzdaten}
\newacronym{RFC}{RFC}{Request for comment. Standards bzw. Standardisierungsvorschläge der \acrshort{IETF}}
\newacronym{ITU-T}{ITU-T}{Standardisierungsausschus der internationalen Fernmelde-Union (International Telecommunication Union, ITU)}
\newacronym{UM}{UM}{Unified Communications. Plattformen die verschiedene Kommunikationsdienste auf einmal bereitstellen}
\newacronym{PSTN}{PSTN}{Public Switched Telephone Network / Öffentliches Fernemeldenetz. Oftmals synonym zu \acrshort{ISDN} verwendet}
\newacronym{DDoS}{DDoS}{Distributed Denial Of Service- Angriffsmethode, bei der von vielen verteilten Endpunkten ein Angriffsziel gezielt überlastet wird}
\newacronym{MITM}{MITM}{Man in the middle. Angriffsmethode, bei der Kommuikationsverbindungen zwischen zwei Teilnehmern von Dritten abgefangen / abgehört werden}
\newacronym{IDS}{IDS}{Intrusion Detection System}
\newacronym{PBX}{PBX}{Private Branch Exchange, englische Bezeichnung für Telefonanlage}
\newacronym{TLS}{TLS}{Transport Layer Security}
\newacronym{Zertifikat}{Zertifikat}{Ein digitales Zertifikat zur \acrshort{TLS}-Verschlüsselung gemäß dem gemäß X.509 Standard der \acrshort{ITU-T}. Vgl.\url{https://www.itu.int/rec/T-REC-X.509-201910-I}}
\newacronym{HTTPS}{HTTPS}{Hyper-Text-Transport-Protocol-Secure}
\newacronym{ARP}{ARP}{Address Resolution Protocol}
\newacronym{MAC}{MAC}{Media Access Control}
\newacronym{TKA}{TKA}{Telekommunikationsanlage / Telefonanlage}
\newacronym{IRC}{IRC}{Internet Relay Chat, ein textbasierter Server-Client-Chat gemäß RFC1459. Vgl. \url{https://tools.ietf.org/pdf/rfc1459.pdf}}
\newacronym{H.323}{H.323}{VoIP-Protokollstapel gemäß Spezifikationen der \acrshort{ITU-T}}
\newacronym{RTP}{RTP}{Real Time Protocol, ein Protokoll zur Ende-zu-Ende-Medienübertragung über paketbasierte Netze gemäß RFC 3550. Vgl. \url{https://tools.ietf.org/pdf/rfc3550.pdf}}
\newacronym{SRTP}{SRTP}{Verschlüsselte Version von \acrshort{RTP} gemäß RFC 3711. Vgl. \url{https://tools.ietf.org/pdf/rfc3711.pdf}}
\newacronym{SBC}{SBC}{Session Border Controller, Application Layer Gateway für VoIP-Anwendungen}
\newacronym{ALG}{ALG}{Application Layer Gateway}
\newacronym{Transcodierung}{Transcodierung}{Der Vorgang der Umwandlung eines Medienstroms in einen zweiten mit anderen Eigenschaften z.B. für Paketgröße, Komprimierung etc.}
\newacronym{VLAN}{VLAN}{Virtual Local Area Network}
\newacronym{VPN}{VPN}{Virtual Private Network}
\newacronym{SQL}{SQL}{Structured Query Language. Strukturierte Abfragesprache für (relationale) Datenbanken}
\newacronym{LDAP}{LDAP}{Lightweight Directory Access Protocol. Ein Protokoll zum Zugriff auf dateibasierte Verzeichnisdienste, ursprünglich in RFC 4511 spezifiziert. Vgl. \url{https://tools.ietf.org/pdf/rfc4511.pdf}}
\newacronym{B2BUA}{B2BUA}{Back-to-Back-User-Agent. Bezeichnung für eine \acrshort{SIP}-spezifische Komponente, die gleichzeitig Server- und Client ist}
\newacronym{Spoofing}{Spoofing}{Sinngemäß zu etwas verfälschen bzw. etwas absichtlich fälschen}
\newacronym{Spoofen}{Spoofen}{Vgl. \acrshort{Spoofing}}
\newacronym{RADIUS}{RADIUS}{Remote Authentication Dial-In User Service, eine Möglichkeit zur Authentifizierung von Netzwerkteilnehmern gemäß IEEE 802.1x}
\newacronym{OSI/ISO-Referenzmodell}{OSI/ISO-Referenzmodell}{Hierarchisches, siebenschichtiges Modell, dass die logische Struktur eines Netzwerks beschreibt. Spezifiziert als X.200-Standard der ITU-T.}
\newacronym{Registrar}{Registrar}{Komponente eines \acrshort{VoIP}-Netzes, die aktive Verbindungen mit einem Nutzerverzeichnis verknüpft und somit Lokationsdienste ermöglicht (z.B. Authentifizierung eines Teilnehmers)}

% Abkürzungen werden nur ausgegeben, wenn sie auch verwendet wurden.
% Mit title=Abkürzungsverzeichnis kann die Bezeichnung gesetzt werden
 \printglossary[type=\acronymtype,title=Abkürzungsverzeichnis und Glossar]
 
% Neue Seite
\newpage