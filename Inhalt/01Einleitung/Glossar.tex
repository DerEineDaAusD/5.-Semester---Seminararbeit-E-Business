\begin{comment}
Ein kombiniertes Fachwort- und Abkürzungsverzeichnis auf Basis des Paketes "glossaries", welches Akronyme und Fachbegriffe in einem Kapitel nacheinander ausgibt.
Nur eine Überschrift wird ins Inhaltsverzeichnis aufgenommen.
Lösung gemäß https://www.overleaf.com/learn/latex/glossaries
Abkürzungen können im Text mit \acrshort{} ausgegeben werden
Fachbegriffe mit \GLs - Wichtig: Großes G!
\end{comment}

%--------------------------------------------------------------------%
% In diesem Abschnitt werden die Abkürzungen mit \newacronym definiert
%---------------------------------------------------------------------%
\newacronym{EDV}{EDV}{Elektronische Datenverarbeitung}
\newacronym{POC}{POC}{Proof Of Concept}
\newacronym{HA}{HA}{High Aviability / Hochverfügbarkeit}
\newacronym{IaaS}{IaaS}{Infrastructure as a service}
\newacronym{SaaS}{SaaS}{Software as a Service}
\newacronym{OS}{OS}{Operating Systems / Betriebssystem}
\newacronym{VMM}{VMM}{Virtual Machine Manager}
\newacronym{VM}{VM}{Virtual Machine / Virtuelle Maschine}
\newacronym{CE}{CE}{Containerengine / Container-Engine}
\newacronym{CPU}{CPU}{Central Processing Unit / Zentrale Recheneinheit (eines Computers)}
\newacronym{RAM}{RAM}{Random Access Memory / Wahlfreier Zugriffsspeicher}
\newacronym{IoT}{IoT}{Internet of Things / Internet der Dinge}
\newacronym{UFS}{UFS}{Union File System}
\newacronym{AWS}{AWS}{Amazon Web Services}
\newacronym{ITU}{ITU}{International Telecommunication Union}
\newacronym{SIP}{SIP}{Session Initiation Protocol}
\newacronym{UAC}{UAC}{User Account Control / Nutzerkontenkontrolle}
\newacronym{CA}{CA}{Certificate Authority / Zertifizierungstelle}
\newacronym{STUN}{STUN}{Session Traversal Utilities for NAT}
\newacronym{TURN}{TURN}{Traversal Using Relays around NAT}
\newacronym{NAT}{NAT}{Network Address Translation}
\newacronym{ITU-T}{ITU-T}{International Telecommunication Union}
\newacronym{VoIP}{VoIP}{Voice-over-IP}
\newacronym{BSI}{BSI}{Bundesamt für Sicherheit in der Informationstechnik}
\newacronym{VAF}{VAF}{Bundesverband Telekommunikation e.V}
\newacronym{QoS}{QoS}{Quality of Service}
\newacronym{IP}{IP}{Internet Protocol}
\newacronym{IPv4}{IPv4}{Internet Protocol V4}
\newacronym{IPv6}{IPv6}{Internet Protocol V6}
\newacronym{Payload}{Payload}{Nutzdaten}
\newacronym{UM}{UM}{Unified Communications}
\newacronym{PSTN}{PSTN}{Public Switched Telephone Network / Öffentliches Fernemeldenetz}
\newacronym{IDS}{IDS}{Intrusion Detection System}
\newacronym{PBX}{PBX}{Private Branch Exchange, englische Bezeichnung für Telefonanlage}
\newacronym{TLS}{TLS}{Transport Layer Security}
\newacronym{HTTPS}{HTTPS}{Hyper Text Transport Protocol Secure}
\newacronym{ARP}{ARP}{Address Resolution Protocol}
\newacronym{MAC}{MAC}{Media Access Control}
\newacronym{TKA}{TKA}{Telekommunikationsanlage / Telefonanlage}
\newacronym{H.323}{H.323}{VoIP-Protokollstapel gemäß Spezifikationen der \acrshort{ITU-T}}
\newacronym{SBC}{SBC}{Session Border Controller}
\newacronym{ALG}{ALG}{Application Layer Gateway}
\newacronym{VLAN}{VLAN}{Virtual Local Area Network}
\newacronym{VPN}{VPN}{Virtual Private Network}
\newacronym{SQL}{SQL}{Structured Query Language}
\newacronym{B2BUA}{B2BUA}{Back-To-Back-User-Agent, vgl. \GLs{Back-to-Back-User-Agent}}
\newacronym{GUI}{GUI}{Graphical User Interface}
\newacronym{DNS}{DNS}{Domain Name System}
\newacronym{URL}{URL}{Uniform Ressource Locator}
\newacronym{SDN}{SDN}{Software-defined Network}
\newacronym{LAN}{LAN}{Local area network}
\newacronym{WAN}{WAN}{Wide area network}
\newacronym{IEEE}{IEEE}{Institute of Electrical and Electronics Engineers}
\newacronym{API}{API}{Application Programming Interface}
\newacronym{MPLS}{MPLS}{Multiprotocol Label Switching}
\newacronym{BGP}{BGP}{Border Gateway Protocol}
\newacronym{FCS}{FCS}{Frame Check Sequence}
\newacronym{TCP}{TCP}{Transport Control Protocol}
\newacronym{UDP}{UDP}{User Datagramm Protocol}
\newacronym{TRILL}{TRILL}{Transparent Interconnection of Lots of Links}
\newacronym{DSL}{DSL}{Digital Subscriber Line}
\newacronym{NFV}{NFV}{Network Function Virtualisation}
\newacronym{LTE}{LTE}{Long Term Evolution}
\newacronym{WLAN}{WLAN}{Wireless Local Area Network}
\newacronym{XML}{XML}{Extensible Markup Language}
\newacronym{JSON}{JSON}{JavaScript Object Notation}
\newacronym{YANG}{YANG}{Yet Another Next Generation}
\newacronym{WAN}{WAN}{Wide Area Network}
\newacronym{OSS}{OSS}{Operational Support System}
\newacronym{ITIL}{ITIL}{IT Infrastructure Libary}
\newacronym{COBIT}{COBIT}{Control Objectives for Information and Related Technology}
\newacronym{KM}{KM}{Knowledge Management}
\newacronym{CRM}{CRM}{Customer Relationship Management}
\newacronym{ERP}{ERP}{Enterprise Ressource Planning}
\newacronym{DMS}{DMS}{Document Management System}
\newacronym{OLAP}{OLAP}{Online Analytical Processing}
\newacronym{DSS}{DSS}{Decision Support System}
\newacronym{WYSIWYG}{WYSIWYG}{What You See Is What You Get}
\newacronym{CTI}{CTI}{Computer Telephony Integration}
\newacronym{UC}{UC}{Unified Communications}
\newacronym{EZB}{EZB}{Europäische Zentralbank}
\newacronym{BaaS}{BaaS}{Blockchain as a Service}
\newacronym{XaaS}{XaaS}{Anything as a Service}
\newacronym{PaaS}{PaaS}{Plattform as a Service}
\newacronym{KMU}{KMU}{Kleine und mittelständische Unternehmen}
\newacronym{ITK}{ITK}{Informations- und Telekommunikation(sbranche)}
\newacronym{BI}{BI}{Business Intelligence}
\newacronym{IT}{IT}{Informationstechnologie}
\newacronym{FIS}{FIS}{Führungsinformationssystem}
\newacronym{ETL}{ETL}{Extract - Transform - Load}
\newacronym{EVA}{EVA}{Eingabe - Verarbeitung - Ausgabe (-Prinzip)}
\newacronym{hdfs}{hdfs}{Hadoop File System}
\newacronym{KI}{KI}{Künstliche Intelligenz}
\newacronym{AI}{AI}{Artificial Intelligence}
\newacronym{RMSE}{RMSE}{Root Mean Squared Error}
\newacronym{k-NN}{k-NN}{k-nearest neighbor (algorithm)}
\newacronym{POSIX}{POSIX}{Portable Operating System Interface}
\newacronym{NIST}{NIST}{National Institute of Standards and Technology}
\newacronym{P2P}{P2P}{Peer-to-Peer}
\newacronym{SCM}{SCM}{Supply Chain Management}
\newacronym{QLDB}{QLDB}{Amazon Quantum Ledger Database}
\newacronym{GBE}{GBE}{German Blockchain Ecosystem}
\newacronym{SDK}{SDK}{Software Development Kit}
\newacronym{CI}{CI}{Continuous Integration}
\newacronym{CD}{CD}{Continuous Development}
\newacronym{XP}{XP}{Extreme Programming}
\newacronym{USP}{USP}{Unique Selling Proposition / Alleinstellungsmerkmal}
\newacronym{IDE}{IDE}{Interactive Development Enviroment / Interaktive Entwicklungsumgebung}

%--------------------------------------------------------------------------%
% In diesem Abschnitt werden die Fachbegriffe mit \newglossaryentry definiert
%--------------------------------------------------------------------------%
\newglossaryentry{OSI/ISO-Referenzmodell}
{
    name=OSI/ISO-Referenzmodell,
    description={Hierarchisches, siebenschichtiges Modell, das die logische Struktur eines Netzwerks beschreibt. Spezifiziert als X.200-Standard der \acrshort{ITU-T}}
}
\newglossaryentry{IETF}
{
    name=IETF,
    description={Internet Engineering Task Force, eine offene Gesellschaft zur Schaffung von Standards im Internet. Vgl. \url{https://ietf.org/about/}}
}
\newglossaryentry{ISDN}
{
    name=ISDN,
    description={Integrated Systems Digital Network / Dienstintegrierendes Digitales Netzwerk}
}
\newglossaryentry{RFC}
{
    name=RFC,
    description={Request for comment. Standards bzw. Standardisierungsvorschläge der IETF}
}
\newglossaryentry{ITU-T}
{
    name=ITU-T,
    description={Standardisierungsausschus der internationalen Fernmelde-Union (International Telecommunication Union, ITU)}
}
\newglossaryentry{DDoS}
{
    name=DDoS,
    description={Distributed Denial Of Service- Angriffsmethode, bei der von vielen verteilten Endpunkten ein Angriffsziel gezielt überlastet wird}
}
\newglossaryentry{MITM}
{
    name=MITM,
    description={Man in the middle. Angriffsmethode, bei der Kommuikationsverbindungen zwischen zwei Teilnehmern von Dritten abgefangen / abgehört werden}
}
\newglossaryentry{Zertifikat}
{
    name=MITM,
    description={Ein digitales Zertifikat zur \acrshort{TLS}-Verschlüsselung gemäß dem gemäß X.509 Standard der \acrshort{ITU-T}. Vgl.\url{https://www.itu.int/rec/T-REC-X.509-201910-I}}
}
\newglossaryentry{IRC}
{
    name=IRC,
    description={Internet Relay Chat, ein textbasierter Server-Client-Chat gemäß RFC1459. Vgl. \url{https://tools.ietf.org/pdf/rfc1459.pdf}}
}
\newglossaryentry{RTP}
{
    name=RTP,
    description={Real Time Protocol, ein Protokoll zur Ende-zu-Ende-Medienübertragung über paketbasierte Netze gemäß RFC 3550. Vgl. \url{https://tools.ietf.org/pdf/rfc3550.pdf}}
}
\newglossaryentry{SRTP}
{
    name=SRTP,
    description={Verschlüsselte Version von \acrshort{RTP} gemäß RFC 3711. Vgl. \url{https://tools.ietf.org/pdf/rfc3711.pdf}}
}
\newglossaryentry{Transcodierung}
{
    name=Transcodierung,
    description={Der Vorgang der Umwandlung eines Medienstroms in einen zweiten mit anderen Eigenschaften z.B. für Paketgröße, Komprimierung etc}
}
\newglossaryentry{LDAP}
{
    name=LDAP,
    description={Lightweight Directory Access Protocol. Ein Protokoll zum Zugriff auf dateibasierte Verzeichnisdienste, ursprünglich in RFC 4511 spezifiziert. Vgl. \url{https://tools.ietf.org/pdf/rfc4511.pdf}}
}
\newglossaryentry{Back-to-Back-User-Agent}
{
    name=Back-to-Back-User-Agent,
    description={Back-to-Back-User-Agent (B2BUA). Bezeichnung für eine \acrshort{SIP}-spezifische Komponente, die gleichzeitig Server- und Client ist}
}
\newglossaryentry{Spoofing}
{
    name=Spoofing,
    description={Auch Spoofen. Sinngemäß zu etwas verfälschen bzw. etwas absichtlich fälschen}
}
\newglossaryentry{RADIUS}
{
    name=RADIUS,
    description={Remote Authentication Dial-In User Service, eine Möglichkeit zur Authentifizierung von Netzwerkteilnehmern gemäß IEEE 802.1x}
}
\newglossaryentry{Registrar}
{
    name=Registrar,
    description={Komponente eines \acrshort{VoIP}-Netzes, die aktive Verbindungen mit einem Nutzerverzeichnis verknüpft und somit Lokationsdienste ermöglicht (z.B. Authentifizierung eines Teilnehmers)}
}
\newglossaryentry{URL}
{
    name=URL,
    description={Uniform Ressource Locator. Von der \gls{IETF} spezifizierter Standard zur Adressierung von Ressourcen im Internet. Vgl. \url{https://tools.ietf.org/html/rfc1738}}
}
\newglossaryentry{Load-Balancing}
{
    name=Load-Balancing,
    description={Bezeichnet die Verlagerung von Anfragen/Berechnungen auf weniger ausgelastete Komponenten}
}
\newglossaryentry{Kernel}
{
    name=Kernel,
    description={Bezeichnet den eigentlichen Kern eines Betriebssystems}
}
\newglossaryentry{Hypervisor}
{
    name=Hypervisor,
    description={Bezeichnet eine Komponente zur Überwachung und Verwaltung virtueller Maschinen}
}
\newglossaryentry{Kernelisolierung}
{
    name=Kernelisolierung,
    description={Bezeichnet ein (Sicherheits-)Konzept, bei dem der Betriebssystemkern (Kernel) von normalen Anwendungen abgeschirmt bzw. isoliert wird, um zugriffe gewöhnlicher Anwendungen direkt auf das Betriebssystem zu vermeiden}
}
\newglossaryentry{Overhead}
{
    name=Overhead,
    description={Bezeichnet ungenutze Ressourcen bzw. unnötigen Administrations- und Verwaltungsaufwand der bei wachsender Systemgröße entsteht. Ist in der Bedeutung synonym zu z.B. Wasserkopf}
}
\newglossaryentry{Benchmarks}
{
    name=Benchmarks,
    description={Bezeichnet die Ermittlung von Kennzahlen eines Systems, meist zur Performance- und Leistungsbeurteilung von Hardware}
}
\newglossaryentry{Node}
{
    name=Node,
    description={Bezeichnet im Kontext der Containerisierung immer eine vollständige Container-Installation mit eigener Container-Engine}
}
\newglossaryentry{Engine}
{
    name=Engine,
    description={Bezeichnet den Kern einer Containerisierungslösung, den eigentlichen Container-Host}
}
\newglossaryentry{Daemon}
{
    name=Daemon,
    description={Bezeichnet einen permanent laufenden und verfügbaren Systemprozess. Unter Windows als Dienst bezeichnet}
}
\newglossaryentry{Clustering}
{
    name=Clustering,
    description={Bezeichnet die logische Bündelung von Hardware- oder Softwaresystemen für Zwecke wie Leistungserhöhung oder Ausfallsicherheit}
}
\newglossaryentry{AJAX}
{
    name=AJAX,
    description={Ajax ist ein Konzept der asymmetrischen (versetzten) Dateiübertragung zwischen einem (Web-)Server und einem Browser. AJAX ermöglicht es, Bestandteile einer Seite neu zu laden, ohne das dabei gleich die gesamte angezeigte Seite neu geladen werden muss}
}
\newglossaryentry{MSI}
{
    name=MSI,
    description={Ein \acrshort{MSI}-Paket (Microsoft Installer File) ist ein Softwareinstallationspaket für Windows-Systeme, welches benutzerdefinierte Voreinstellungen enthalten kann, die beim Installationsvorgang automatisch eingerichtet werden}
}
\newglossaryentry{Subnetz}
{
    name=Subnetz,
    description={Ein Subnetz ist ein innerhalb eines IP-Bereiches getrennt adressierbarer Bereich des Internet-Protocols}
}
\newglossaryentry{ZIP}
{
    name=ZIP,
    description={Ein Dateiformat zur verlustfreien Komprimierung von Daten}
}
\newglossaryentry{Codec}
{
    name=Codec,
    description={Als Codec (Zusammengesetzt "coder" und "decoder") bezeichnet man ein Algorithmenpaar, das Daten oder Signale digital kodiert und dekodiert. Ein Codec wie z.B. g711a enthält Informationen darüber, mit welcher Abtastrate, Häufigkeit etc. ein Signal aufgezeichnet wurde}
}
\newglossaryentry{1st Level-Support}
{
    name=1st Level-Support,
    description={Erste Ebene des Anwendersupports, an welche sich ein Benutzer direkt wenden kann}
}
\newglossaryentry{CMD}
{
    name=CMD,
    description={cmd.exe, Kommandozeileninterpreter des Windows-Betriebssystems (Offiziell als Windows-Eingabeaufforderung bezeichnet)}
}
\newglossaryentry{paketvermittelt}
{
    name=paketvermittelt,
    description={In einem paketvermittelt Datennetz werden Informationen in (Teil-)Stücken übertragen. Abhängig von den Übertragungseigenschaften des Netzes werden unterschiedlich viele Pakete erstellt und gesendet, die anschließend vom Empfänger wieder zusammengesetzt werden}
}
\newglossaryentry{Orchestrar}
{
    name=Orchestrar,
    description={Steuerungskomponente zur Verwaltung einer virtuellen Umgebung. Der Orchestrar erzeugt, löscht, startet und stoppt virtuelle Maschinen, entweder manuell oder automatisiert}
}
\newglossaryentry{Bursts}
{
    name=Bursts,
    description={Kurze, sprungartige ad-hoc Datenübertragungen}
}
\newglossaryentry{DevOps}
{
    name=DevOps,
    description={Kunstwort aus Development und (IT) Operations. Prozessansatz für die direkte Kombination von Softwareentwicklung und IT-Systemadministration.  Beschreibt einen Ansatz, bei dem durch gemeinsame Zusammenarbeit von Entwicklung und Administration die Geschwindigkeit der Auslieferung von Software und die Qualität der erzeugten Software verbessert werden sollen}
}
\newglossaryentry{OpenFlow}
{
    name=OpenFlow,
    description={OpenFlow ist ein Standard von der Open Network Foundation (\url{https://www.opennetworking.org/sdn-definition/}) zum Routing von Datenpaketen in einem software-defined Network}
}
\newglossaryentry{NETCONF}
{
    name=NETCONF,
    description={NETCONF (Network Configuration Protocol) ist ein zuerst mit RFC 4741 (\url{https://tools.ietf.org/html/rfc4741}) entworfenes und mit RFC 6241 (\url{https://tools.ietf.org/html/rfc6241}) erweiteres Netzwerkprotokoll zur Konfiguration und Verwaltung von Netzwerkkomponenten auf Basis einer XML-Syntax}
}
\newglossaryentry{Regression}
{
    name=Regression,
    description={Technik der Statistik zur Modellierung linearer Zusammenhänge zwischen einer abhängigen und einer oder mehreren unabhängigen Variablen. Regression kann dabei quantitativ beschreibend oder für zukünftige Werte prognostizierend sein}
}
\newglossaryentry{Data Warehouse}
{
    name=Data Warehouse,
    description={Begriff der Business Intelligence, welcher einen Teil der IT-Infrastruktur bezeichnet, in welchem Daten für BI-Analysen aus definierten Quellen gesammelt, aufbereitet und für die weitere Verarbeitung bereitgestellt werden. Die Datenquellen sind dabei zumeist bekannt und vollständig oder mindestens semi-strukturiert}
}
\newglossaryentry{Data Lake}
{
    name=Data Lake,
    description={Begriff der Big Data, welcher einen Teil der IT-Infrastruktur bezeichnet, in welchem Daten für Big Data-Analysen aus unterschiedlichsten Quellen gesammelt, aufbereitet und für die weitere Verarbeitung bereitgestellt werden. Im Gegensatz zum Data Warehouse werden beliebige Daten gesammelt. Die Daten können dabei unstrukturiert, semi-strukturiert oder vollständig strukturiert sein}
}
\newglossaryentry{Big Data}
{
    name=Big Data,
    description={bezeichnet große Mengen Daten aus unterschiedlichsten Quellen, wie Unternehmen, Sensoren, dem Internet, Social Media usw. Diese gewaltigen Datenmengen liegen in solchen Größenordnungen und unterschiedlichsten Formen und Formaten vor, dass ihre Auswertung zur Informations- und Wissensgewinnung eigene Techniken und Technologien benötigt}
}
\newglossaryentry{Data Mart}
{
    name=Data Mart,
    description={sind Teil eines Data Warehouses. Während das Data Warehouse den gesamten Datenbestand enthält, handelt es sich bei Data Marts um (Teil-)Datenbanken spezifisch für (Fach-)Abteilungen oder bestimmte Fragestellungen / Anforderungen}
}
\newglossaryentry{Data Mining}
{
    name=Data Mining,
    description={Data Mining bezeichnet den Vorgang der explorativen Analyse eines bestehenden Datenbestandes. Mittels systematischer (statistischer) Analyse sollen unterschiedlichste Informationen aus einem vorliegenden Datenbestand gewonnen werden}
}
\newglossaryentry{Homoskedastizitaet}
{
    name=Homoskedastizitaet,
    description={Homoskedastizität beschreibt in der linearen Regression die konstante Varianz der Residuen über die betrachtete Datenmenge. Die Abweichungen müssen konstant bzw. normalverteilt sein}
}
\newglossaryentry{Multikollinearitaet}
{
    name=Multikollinearitaet,
    description={Bezeichnet in der linearen Regression die Korrelation zwischen zwei oder mehr der erklärenden Variablen (Regressoren). Multikollinearität oder Auto-Korrelation erschwert die Bestimmung der tatsächlich relevanten Einflussfaktoren auf ein Regressionsmodell}
}
\newglossaryentry{on-premise}
{
    name=on-premise,
    description={Bezeichnet den Betrieb von (IT-)Infrastruktur im eigenen Unternehmen (\glqq{}auf eigenem Grund\grqq{}). Eine on-premise betriebene Komponente befindet sich völlig im Besitzt des Betreibers}
}
\newglossaryentry{off-premise}
{
    name=off-premise,
    description={Bezeichnet den Betrieb von (IT-)Infrastruktur außerhalb des eigenen Unternehmens. Eine off-premise betriebene Komponente wird durch Dritte bereitgestellt und / oder für den Eigentümer betreut}
}
\newglossaryentry{serverless}
{
    name=serverless,
    description={Hostingform, bei welcher Ressourcen erst auf Anforderung durch den Dienstnutzer beim Anbieter zugewiesen werden. Der Kunde greift nicht auf permanent für ihn verfügbare Ressourcen (Server) zu, sondern erhält Ressourcen erst auf Anforderung, z.B. bei Ausführung einer Anweisung}
}
\newglossaryentry{event-driven}
{
    name=event-driven,
    description={Event-driven Computing bezeichnet die Bereitstellung von Ressourcen (z.B. Rechenleistung) erst auf Anfordung, d.h. Auslösung durch ein Event. Ein Event kann ein Automatismus sein, z.B. ein API-Zugriff. Vgl. auch \glqq{}\gls{serverless} Computing\grqq{}}
}
\newglossaryentry{Peer-to-Peer}
{
    name=Peer-to-Peer,
    description={Bezeichnet einen Netzwerktyp, bei den Verbindungen direkt zwischen den Teilnehmern aufgebaut werden. Alle Teilnehmer sind gleichberechtigt, es gibt keine zentrale Instanz zur Steuerung des Netzwerks}
}
\newglossaryentry{Proof-of-Work}
{
    name=Proof-of-Work,
    description={Bezeichnung für ein Bestätigungsverfahren innerhalb von Blockchains, bei denen eine Aktion (z.B. Transaktion) von einem Teilnehmer über einen mit Aufwand / Leistung verbundenen Nachweis bestätigt wird. Dieser Nachweis wird in Form von Erledigung bestimmter Aufgaben erbracht, z.B. dem Lösen kryptografischer Rätsel}
}
\newglossaryentry{Asymmetrische} % Workaround, da nur ASCII-/ANSI-Zeichen bei \gls zulässig
{
    name=Asymmetrische Verschlüsselung,
    description={Bezeichnet ein Verschlüsselungsverfahren, bei dem die beteiligten Parteien keinen gemeinsamen Schlüssel benötigen. Stattdessen erfolgen Ver- und Entschlüsselung von jedem Kommunikationspartner mit zwei unterschiedlichen Schlüsseln, einem öffentlichen (public) und einem privaten Schlüssel (Key). Daten, welche mit dem öffentlichen Schlüssel eines Kommunikationsteilnehmers verschlüsselt sind, können mit dessen privatem Key wieder entschlüsselt werden. Asymmetrische Verschlüsselung beruht damit auf (angenommenen) nicht umkehrbaren Funktionen und bietet im Gegensatz zu symmetrischer Verschlüsselung den Vorteil auf den Schlüsselaustausch zwischen den Kommunikationspartnern (über ein potenziell unsicheres Medium) zu verzichten. Weiterführende Informationen siehe \url{http://ddi.cs.uni-potsdam.de/Lehre/e-commerce/elBez2-5/page06.html\#}}
}
\newglossaryentry{Hash-Verfahren}
{
    name=Hash-Verfahren,
    description={Hash-Funktionen (von engl. \glqq{}to hash\grqq{} = zerhacken, dt. \textit{Streuwertfunktion}) sind mathematische Verfahren, die eine bestimmte Eingabemenge (z.B. Zahlen oder Wörter) auf eine bestimmte Zielmenge (den s.g. Hashwert) abbilden. Hashfunktionen sollen dabei für jede unterschiedliche Eingabe einen unterschiedlichen Hashwert liefern (was allerdings nicht immer möglich ist, vgl. die s.g. Kollisionen. Kollision bedeutet, dass zwei unterschiedliche Eingaben den gleichen Hashwert erzeugen).In der Kryptografie wird diese Eigenschaft im Rahmen eines Einwegverfahrens zur Signatur und Validierung benutzt: Würde die Eingabe verändert (z.B. absichtlich verfälscht), muss sich auch der resultierende Hashwert ändern und eine Veränderung ist erkennbar. Weiterführende Informationen siehe \url{https://www.nm.ifi.lmu.de/teaching/Vorlesungen/2015ws/itsec/_skript/itsec-k9-v11.0.pdf}}
}
\newglossaryentry{Merkle-Hashs}
{
    name=Merkle-Hashs,
    description={Merkle-Hashs sind aufeinander baumförmig aufbauende Hash-Funktionen, die auf ein Patent von Ralph C. Merkle zurückgehen. Besonderheit dieser \glqq{}Hash-Trees\grqq{} ist, das jeder Zweig in der Struktur durch Kenntnis des Wurzel-Hashs (\glqq{}root hahs\grqq{}) unabhängig von anderen Zweigen und auch bei Nichtvorliegen von Teilen der Struktur geprüft werden kann. Somit kann über einen Teil die gesamte Struktur validiert werden. Blockchains nutzen dieses Verfahren u.a. zur Überprüfung und Validierung ineinander geschachtelter Teilschritte (Transaktionen)}
}
\newglossaryentry{Vendor-Lockin}
{
    name=Vendor-Lockin,
    description={Bindung an bzw. Abhängigkeit von einem Anbieter verbunden mit Aufwand bzw. Hürden aufgrund derer z.B. ein Produkt nicht einfach zu einem konkurrierenden Anbieter verlagert werden kann}
}
\newglossaryentry{SCRUM}
{
    name=SCRUM,
    description={(von engl. \glqq{}Gedränge\grqq{}) ist ein Vorgehensmodell für agile Softwareentwicklun und Projektmanagement. SCRUM etabliert die Grundideen des agilen Manifests (\url{https://agilemanifesto.org/}), Ziel von SCRUM ist es, einem (Softwareentwicklungs-) Team notwendige Freiheiten und Freiräume zur selbstständigen Erreichung vorgegebener Ziele zu verschaffen (anstelle fest vorgebender Abläufe und Zuständigkeiten). Weiterführende Informationen unter \url{https://www.scrumalliance.org/}}
}
\newglossaryentry{Time-to-Market}
{
    name=Time-to-Market,
    description={bezeichnet die Zeitspanne zwischen der Entstehung einer Produktidee oder eines Serviceangebotes bis zur Erreichung der Marktreife. Definition und weiterführende Informationen vgl. \url{https://wirtschaftslexikon.gabler.de/definition/time-market-54271/version-277318}}
}


% Abkürzungen werden nur ausgegeben, wenn sie auch verwendet wurden.
% Mit title=Abkürzungsverzeichnis kann die Bezeichnung gesetzt werden
\printglossary[type=\acronymtype,title=Abkürzungsverzeichnis]

% Seitenumbruch zwischen Abkürzungen und Fachbegriffen
\newpage

% Fachbegriffe ausgeben
\printglossary[type=main,title=Fachbegriffe]