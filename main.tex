% LaTeX Sourcode

% Präambel
% Parameter von \documentclass
% https://texblog.org/2013/02/13/latex-documentclass-options-illustrated/
\documentclass[12pt,pdftex,parskip=half]{article}

% Diese Datei enthält sämtliche zusätzlich eingebundenen Pakete

% URLs mit Zeilenumbruch ermöglichen
% \usepackage[hyphens]{url}

% PDF Dokumenteneigenschaften - Hyperref ermöglicht Meta-Informationen in der PDF
% https://www.namsu.de/Extra/pakete/Hyperref.html
% Links werden im Dokument schwarz dargestellt
\usepackage[pdftitle={Seminararbeit Containerisierung vs. Virtualisierung},pdfsubject={Virtualisierung und Containerisierung},pdfauthor={Danny Langenbach},pdfkeywords={VoIP, Security, Risiken,SBC},pdfstartview=FitH,colorlinks=true,linkcolor=black,urlcolor=black,citecolor=black]{hyperref}

% https://tex.stackexchange.com/questions/3033/forcing-linebreaks-in-url
% https://tex.stackexchange.com/a/3034
% https://tex.stackexchange.com/a/10419
\PassOptionsToPackage{hyphens}{url}\usepackage{hyperref}

% Komplette PDFs einbinden
\usepackage{pdfpages}

% main=ngerman legt die Hauptsprache des Dokuments auf Neue Deutsche Rechtschreibung fest
% (Umlaute, Datum etc)
% englisch erlaubt es einzelne Abschnitte in Englisch zu definieren
\usepackage[main=ngerman, english]{babel}
\usepackage[T1]{fontenc}
\usepackage[utf8]{inputenc}  

% Für das Euro-Symbol
% \usepackage{eurosym}  \DeclareUnicodeCharacter{20AC}{\euro}

% Graphix ermöglicht die Einbindung von .jpeg,.jpg,.png usw.
\usepackage{graphicx}

% Kopzeile und Stil
\usepackage{fancyhdr}

\usepackage{caption}
% Fußnote
\usepackage{footmisc}

% Um das Abbildungsverzeichnis ins Inhaltsverzeichnis aufzunehmen
% Nottoc verhindert das sich das Inhaltsverzeichnis selbst als Inhalt nennt
% https://tex.stackexchange.com/a/297985
\usepackage[nottoc]{tocbibind}
% https://texblog.org/2013/04/29/latex-table-of-contents-list-of-figurestables-and-some-customizations/#totoc

% Package für Gesamtseitenzahl
\usepackage{lastpage}

% tabularx für Tabellen
\usepackage{tabularx}

% Für Blindtext
\usepackage{lipsum}

% Zitate APA Style - 6. Version
% Muss auskommentiert werden, wenn IEEEtran verwendet werden soll
% \usepackage{apacite}

% Für das Abkürzungsverzeichnis
% Ermöglicht die Verwendung von \acs im Text und gibt nur verwendete Abkürzungen im Glossar aus
% \usepackage[printonlyused,withpage]{acronym}
% printonlyused = Nur verwendete Acronyme angeben
% WithPage = Seite der Abkürzung im Dokument

% nopostdot = Kein Punkt nach der Abkürzung
% nonumberlist = Keine Seitenzahlen ausgeben, wo die Abkürzungen verwendet werden.
% toc = true nimmt Eintrag ins Inhaltsverzeichnis auf, ohne Notwendigkeit von \addcontentsline
% siehe https://tex.stackexchange.com/a/156742
% \usepackage[acronym,toc=true]{glossaries}
\usepackage[acronym,toc=true]{glossaries}

% \usepackage{xparse}

% Für mehrzeilige Kommentare
\usepackage{verbatim}

% Für Textoverlays wie z.B. "Entwurf" oder "Vertraulich"
% Gestaltung des Wasserzeichens unter /Layout/Wasserzeichen
% \usepackage[firstpage]{draftwatermark} - Wasserzeichen nur auf der ersten Seite
% \usepackage[nostamp]{draftwatermark} - Kein Wasserzeichen irgendwo im Dokument
% \usepackage{draftwatermark}
% Zeilenabstände
%\usepackage[onehalfspacing]{setspace}

% Zeilenabstand global 1.5-zeilig und vorher kein Einzug
\setlength{\parindent}{0pt}  \linespread{1.5}

% Zeilenabstand nach Absatz
\setlength{\parskip}{6pt}
% Kopfzeilenformatierung

% AtBeginDocument = Das globale Layout von Geometry gilt auch für Kopf-  und Fußzeile

\AtBeginDocument{

%\renewcommand{\headrulewidth}{0.4pt}
%\renewcommand{\sectionmark}[1]{\markboth{#1}{}} % set the \leftmark
% \renewcommand*\sectionmarkformat{} % keine Nummerierung im Kopf

%\RenewDocumentCommand{\section}{som}{
%  \IfBooleanTF{#1}
%   {% there's a *
%    \CLASSsection*{#3}\markboth{#3}{}
%   }
%   {% no *
%    \IfNoValueTF{#2}
%     {% no opt arg
%      \CLASSsection{#3}%
%     }
%     {% opt arg
%      \CLASSsection[#2]{#3}
%     }
%   }
% }

% fix \tableofcontents
% \renewcommand{\tableofcontents}{%
%  \section*{\contentsname}%
%  \@starttoc{toc}%
%}

% Rechter Header = leer
% \leftmark = Aktueller Abschnitt
\fancyhead[R]{}

% Mittlerer Header = Aktuelle Seite
\fancyhead[C]{\nouppercase{\thepage}}
%\fancyheadoffset[C]{1cm}

% Linker Header = leer
\fancyhead[L]{}
% \fancyheadoffset[R]{1cm}

}
% Layout der Fußzeile
% Fußzeile ist leer

% Fußzeilenformatierung
% \renewcommand{\footrulewidth}{0.4pt}

% Linke Spalte
\lfoot{} 

% Mittlere Spalte
% \cfoot{\small  \today}
\cfoot{}

% Rechte Spalte 
% \rfoot{\normalfont\small Seite \thepage / \ref{TotPages}}
\rfoot{}
% Seitengestaltung und Abstände
\usepackage[a4paper,left=40mm,right=20mm,bottom=20mm,top=40mm,bindingoffset=0mm,includeheadfoot]{geometry}
% includeheadfoot  = Layout greift auch für Kopf und Fußzeile. 
% includefoot = Nur für Fußzeile
% includehead = Nur für Kopfzeile
%Für Farben im allgemeinen
\usepackage{xcolor} 

% RGB zu CYMK
% https://codebeautify.org/rgb-to-cmyk-converter

%Definition der Firmenfarben
\definecolor{effexxrot}{cmyk}{0.0000,0.9736,0.9163,0.1098}
\definecolor{effexxgrau}{cmyk}{0.0000,0.0100,0.0100,0.6078}
\definecolor{effexxweiß}{cmyk}{0.0000,0.0000,0.0000,0.0000} 

% Definition der FOM-Farben
\definecolor{FOMgruen}{cmyk}{0.9869,0.0000,0.0915,0.4000}

% Verwendung im Text mit \textcloro{Farbenname}{<Textabschnitt>}
% Hier können eigene Header-Format für bestimmte Seiten definiert werden
% Aufruf im Dokument mit \thispagestyle{mystyle}
% Gemäß https://tex.stackexchange.com/questions/519136/set-header-of-a-specific-page-only
% Und Antwort https://tex.stackexchange.com/a/519189

\fancypagestyle{Leitfaden}
{
    % Rechte Seite = manueller Seitentitel
    \fancyhead[R]{Leitfaden}
    
    % Mittlerer Header = Aktuelle Seite
    \fancyhead[C]{\nouppercase{\thepage}}
    %\fancyheadoffset[C]{1cm}

    % Linker Header = leer
    \fancyhead[L]{}
    
    %
    % Fusszeile
    %
    
    % Layout der Fußzeile
    % Fußzeile ist leer

    % Fußzeilenformatierung
    % \renewcommand{\footrulewidth}{0.4pt}

    % Linke Spalte
    \lfoot{} 

    % Mittlere Spalte
    % \cfoot{\small  \today}
    \cfoot{}

    % Rechte Spalte 
    % \rfoot{\normalfont\small Seite \thepage / \ref{TotPages}}
    \rfoot{}
}

\fancypagestyle{Abbildungsverzeichnis}
{
    % Rechte Seite = manueller Seitentitel
    \fancyhead[R]{Abbildungsverzeichnis}
    
    % Mittlerer Header = Aktuelle Seite
    \fancyhead[C]{\nouppercase{\thepage}}
    %\fancyheadoffset[C]{1cm}

    % Linker Header = leer
    \fancyhead[L]{}
    
    %
    % Fusszeile
    %
    
    % Layout der Fußzeile
    % Fußzeile ist leer

    % Fußzeilenformatierung
    % \renewcommand{\footrulewidth}{0.4pt}

    % Linke Spalte
    \lfoot{} 

    % Mittlere Spalte
    % \cfoot{\small  \today}
    \cfoot{}

    % Rechte Spalte 
    % \rfoot{\normalfont\small Seite \thepage / \ref{TotPages}}
    \rfoot{}
}

\fancypagestyle{Abkuerzungsverzeichnis}
{
    % Rechte Seite = manueller Seitentitel
    \fancyhead[R]{Abkuerzungsverzeichnis}
    
    % Mittlerer Header = Aktuelle Seite
    \fancyhead[C]{\nouppercase{\thepage}}
    %\fancyheadoffset[C]{1cm}

    % Linker Header = leer
    \fancyhead[L]{}
    
    %
    % Fusszeile
    %
    
    % Layout der Fußzeile
    % Fußzeile ist leer

    % Fußzeilenformatierung
    % \renewcommand{\footrulewidth}{0.4pt}

    % Linke Spalte
    \lfoot{} 

    % Mittlere Spalte
    % \cfoot{\small  \today}
    \cfoot{}

    % Rechte Spalte 
    % \rfoot{\normalfont\small Seite \thepage / \ref{TotPages}}
    \rfoot{}
}

\fancypagestyle{Glossar}
{
    % Rechte Seite = manueller Seitentitel
    \fancyhead[R]{Glossar}
    
    % Mittlerer Header = Aktuelle Seite
    \fancyhead[C]{\nouppercase{\thepage}}
    %\fancyheadoffset[C]{1cm}

    % Linker Header = leer
    \fancyhead[L]{}
    
    %
    % Fusszeile
    %
    
    % Layout der Fußzeile
    % Fußzeile ist leer

    % Fußzeilenformatierung
    % \renewcommand{\footrulewidth}{0.4pt}

    % Linke Spalte
    \lfoot{} 

    % Mittlere Spalte
    % \cfoot{\small  \today}
    \cfoot{}

    % Rechte Spalte 
    % \rfoot{\normalfont\small Seite \thepage / \ref{TotPages}}
    \rfoot{}
}

\fancypagestyle{Tabellenverzeichnis}
{
    % Rechte Seite = manueller Seitentitel
    \fancyhead[R]{Tabellenverzeichnis}
    
    % Mittlerer Header = Aktuelle Seite
    \fancyhead[C]{\nouppercase{\thepage}}
    %\fancyheadoffset[C]{1cm}

    % Linker Header = leer
    \fancyhead[L]{}
    
    %
    % Fusszeile
    %
    
    % Layout der Fußzeile
    % Fußzeile ist leer

    % Fußzeilenformatierung
    % \renewcommand{\footrulewidth}{0.4pt}

    % Linke Spalte
    \lfoot{} 

    % Mittlere Spalte
    % \cfoot{\small  \today}
    \cfoot{}

    % Rechte Spalte 
    % \rfoot{\normalfont\small Seite \thepage / \ref{TotPages}}
    \rfoot{}
}

\fancypagestyle{Formelverzeichnis}
{
    % Rechte Seite = manueller Seitentitel
    \fancyhead[R]{Formelverzeichnis}
    
    % Mittlerer Header = Aktuelle Seite
    \fancyhead[C]{\nouppercase{\thepage}}
    %\fancyheadoffset[C]{1cm}

    % Linker Header = leer
    \fancyhead[L]{}
    
    %
    % Fusszeile
    %
    
    % Layout der Fußzeile
    % Fußzeile ist leer

    % Fußzeilenformatierung
    % \renewcommand{\footrulewidth}{0.4pt}

    % Linke Spalte
    \lfoot{} 

    % Mittlere Spalte
    % \cfoot{\small  \today}
    \cfoot{}

    % Rechte Spalte 
    % \rfoot{\normalfont\small Seite \thepage / \ref{TotPages}}
    \rfoot{}
}

\fancypagestyle{Symbolverzeichnis}
{
    % Rechte Seite = manueller Seitentitel
    \fancyhead[R]{Symbolverzeichnis}
    
    % Mittlerer Header = Aktuelle Seite
    \fancyhead[C]{\nouppercase{\thepage}}
    %\fancyheadoffset[C]{1cm}

    % Linker Header = leer
    \fancyhead[L]{}
    
    %
    % Fusszeile
    %
    
    % Layout der Fußzeile
    % Fußzeile ist leer

    % Fußzeilenformatierung
    % \renewcommand{\footrulewidth}{0.4pt}

    % Linke Spalte
    \lfoot{} 

    % Mittlere Spalte
    % \cfoot{\small  \today}
    \cfoot{}

    % Rechte Spalte 
    % \rfoot{\normalfont\small Seite \thepage / \ref{TotPages}}
    \rfoot{}
}

\fancypagestyle{Symbolverzeichnis}
{
    % Rechte Seite = manueller Seitentitel
    \fancyhead[R]{Symbolverzeichnis}
    
    % Mittlerer Header = Aktuelle Seite
    \fancyhead[C]{\nouppercase{\thepage}}
    %\fancyheadoffset[C]{1cm}

    % Linker Header = leer
    \fancyhead[L]{}
    
    %
    % Fusszeile
    %
    
    % Layout der Fußzeile
    % Fußzeile ist leer

    % Fußzeilenformatierung
    % \renewcommand{\footrulewidth}{0.4pt}

    % Linke Spalte
    \lfoot{} 

    % Mittlere Spalte
    % \cfoot{\small  \today}
    \cfoot{}

    % Rechte Spalte 
    % \rfoot{\normalfont\small Seite \thepage / \ref{TotPages}}
    \rfoot{}
}

\fancypagestyle{Sperrvermerk}
{
    % Rechte Seite = manueller Seitentitel
    \fancyhead[R]{Sperrvermerk}
    
    % Mittlerer Header = Aktuelle Seite
    \fancyhead[C]{\nouppercase{\thepage}}
    %\fancyheadoffset[C]{1cm}

    % Linker Header = leer
    \fancyhead[L]{}
    
    %
    % Fusszeile
    %
    
    % Layout der Fußzeile
    % Fußzeile ist leer

    % Fußzeilenformatierung
    % \renewcommand{\footrulewidth}{0.4pt}

    % Linke Spalte
    \lfoot{} 

    % Mittlere Spalte
    % \cfoot{\small  \today}
    \cfoot{}

    % Rechte Spalte 
    % \rfoot{\normalfont\small Seite \thepage / \ref{TotPages}}
    \rfoot{}
}

\fancypagestyle{ConfidentialityClause}
{
    % Rechte Seite = manueller Seitentitel
    \fancyhead[R]{Confidentiality Clause}
    
    % Mittlerer Header = Aktuelle Seite
    \fancyhead[C]{\nouppercase{\thepage}}
    %\fancyheadoffset[C]{1cm}

    % Linker Header = leer
    \fancyhead[L]{}
    
    %
    % Fusszeile
    %
    
    % Layout der Fußzeile
    % Fußzeile ist leer

    % Fußzeilenformatierung
    % \renewcommand{\footrulewidth}{0.4pt}

    % Linke Spalte
    \lfoot{} 

    % Mittlere Spalte
    % \cfoot{\small  \today}
    \cfoot{}

    % Rechte Spalte 
    % \rfoot{\normalfont\small Seite \thepage / \ref{TotPages}}
    \rfoot{}
}

\fancypagestyle{Literatur}
{
    % Rechte Seite = manueller Seitentitel
    \fancyhead[R]{Literatur}
    
    % Mittlerer Header = Aktuelle Seite
    \fancyhead[C]{\nouppercase{\thepage}}
    %\fancyheadoffset[C]{1cm}

    % Linker Header = leer
    \fancyhead[L]{}
    
    %
    % Fusszeile
    %
    
    % Layout der Fußzeile
    % Fußzeile ist leer

    % Fußzeilenformatierung
    % \renewcommand{\footrulewidth}{0.4pt}

    % Linke Spalte
    \lfoot{} 

    % Mittlere Spalte
    % \cfoot{\small  \today}
    \cfoot{}

    % Rechte Spalte 
    % \rfoot{\normalfont\small Seite \thepage / \ref{TotPages}}
    \rfoot{}
}

\fancypagestyle{EhrenwoertlicheErklaerung}
{
    % Rechte Seite = manueller Seitentitel
    \fancyhead[R]{Ehrenwörtliche Erklärung}
    
    % Mittlerer Header = Aktuelle Seite
    \fancyhead[C]{\nouppercase{\thepage}}
    %\fancyheadoffset[C]{1cm}

    % Linker Header = leer
    \fancyhead[L]{}
    
    %
    % Fusszeile
    %
    
    % Layout der Fußzeile
    % Fußzeile ist leer

    % Fußzeilenformatierung
    % \renewcommand{\footrulewidth}{0.4pt}

    % Linke Spalte
    \lfoot{} 

    % Mittlere Spalte
    % \cfoot{\small  \today}
    \cfoot{}

    % Rechte Spalte 
    % \rfoot{\normalfont\small Seite \thepage / \ref{TotPages}}
    \rfoot{}
}

\fancypagestyle{DeclarationInLieuOfOath}
{
    % Rechte Seite = manueller Seitentitel
    \fancyhead[R]{Declaration in lieu of oath}
    
    % Mittlerer Header = Aktuelle Seite
    \fancyhead[C]{\nouppercase{\thepage}}
    %\fancyheadoffset[C]{1cm}

    % Linker Header = leer
    \fancyhead[L]{}
    
    %
    % Fusszeile
    %
    
    % Layout der Fußzeile
    % Fußzeile ist leer

    % Fußzeilenformatierung
    % \renewcommand{\footrulewidth}{0.4pt}

    % Linke Spalte
    \lfoot{} 

    % Mittlere Spalte
    % \cfoot{\small  \today}
    \cfoot{}

    % Rechte Spalte 
    % \rfoot{\normalfont\small Seite \thepage / \ref{TotPages}}
    \rfoot{}
}

% % Gestaltung nach https://texblog.org/2012/02/17/watermarks-draft-review-approved-confidential/
% und http://packages.oth-regensburg.de/ctan/macros/latex/contrib/draftwatermark/draftwatermark.pdf

% Text des Wasserzeichens
% Entwurf
\SetWatermarkText{Entwurf}
% Vertraulich
% \SetWatermarkText{Vertraulich}
% Freigegeben
% \SetWatermarkText{Freigegeben}

% Skalierung - "4" bedeckt ganze Seite
\SetWatermarkScale{4}

% Farbe - Graustufe wählen
% Default = 0.8
\SetWatermarkColor[gray]{0.9}

% Andere Farbe wählen
% \SetWatermarkColor[rgb]{1,0,0}

% Fontsize
% Default = 1.2cm
\SetWatermarkFontSize{2cm}

% Winkel des Wasserzeichens
% Default = 45
\SetWatermarkAngle{45}

%\glsnoexpandfields
\makeglossaries

\begin{document}

% Im gesamten Dokument Kopf- und Fußzeile anzeigen
\pagestyle{fancy}

% URLs auch nach / umbrechen lassen:
\addto\UrlBreaks{\do\/}
\def\do@url@hyp{\do\-}
% URL-Umbrüche anzeigen lassen
\show\UrlBreaks

% Vorwort - Deckblatt, Inhaltsverzeichnis etc.
% Die Gliederung richtet sich nach den allgemeinen FOM-Vorgaben aus 
% https://campus.bildungscentrum.de/nfcampus/dc/3667/LeitfadenZurFormalenGestaltungSeminarAbschlussarbeiten_BCW_Stud_2018_02_21.pdf

% Deckblatt
% Deckblatt ohne Kopf- und Fußzeile
% https://tex.stackexchange.com/questions/23766/suppress-fancy-header-and-footer-on-first-page-only
% Keine Header auf Deckblatt
\thispagestyle{empty}

% Keine Seitenabstände auf Deckblatt
\newgeometry{}

\begin{center}
\vfill{

% Logo
{\vspace{\fill}{\includegraphics[width=0.3\textwidth]{Grafiken/FOM_Logo.png}}}\\
\large{FOM - Hochschule für Ökonomie und Management}\\
\large{Studienzentrum Siegen}\\
\vspace{1cm}

% Titel
{\huge DevOps und CD/CI\\ 
 \large{Eine Übersicht für Ansätze aus der Praxis\\}
}
\vspace{1cm}

% Modul
{\large Seminararbeit im  Modul\\E-Business\\}}
{\large{Wirtschaftsinformatik WS 2020/2021\\}}
{\large{Dozent: Herr Markus Dormann B.Sc.\\}}

\end{center}

% Namen und Abgabedatum
\vspace{1cm}
\begin{tabularx}{\textwidth}[b]{p{5cm} X p{5cm}}
Matrikelnummer: 479141\\
Herr Danny Langenbach\\
Abgabe: \today
\end{tabularx}

% Neue Seite
\newpage

% Römische Kapitelnummerierung
% \setcounter{page}{2}
\pagenumbering{roman}

% Inhaltsverzeichnis
\tableofcontents

% Verwendete Version des Leitfadens zur formalen Gestaltung nennen.
\section*{Leitfaden}
\addcontentsline{toc}{section}{Leitfaden}

% Abschnittspezifische Header mit Abschnittsangabe oben rechts
\thispagestyle{Leitfaden}

Dieses Dokument wurde gemäß dem Leitfaden zur formalen Gestaltung von Seminar- und Abschlussarbeiten der FOM, Hochschule für Ökonomie und Management mit Revisionsstand vom 12. Mai 2020 erstellt. Der Leitfaden ist verfügbar unter \href{https://campus.bildungscentrum.de/nfcampus/dc/4875/LeitfadenZurFormalenGestaltungSeminarAbschlussarbeiten_BCW_Stud_2020_04_29.pdf}{https://campus.bildungscentrum.de}.\newline
Zitiert wird gemäß dem \acrshort{IEEE}-Zitationsstil, beschrieben unter \href{https://ieee-dataport.org/sites/default/files/analysis/27/IEEE\%20Citation\%20Guidelines.pdf}{https://ieee-dataport.org}.

% Neue Seite
\newpage

% Abbildungsverzeichnis hinzufügen
%\section*{Abbildungsverzeichnis}
%\addcontentsline{toc}{section}{Abbildungsverzeichnis}

% Abschnittspezifische Header mit Abschnittsangabe oben rechts
% \thispagestyle{Abbildungsverzeichnis}

% Manuell den Titel = Abbildungsverzeichnis setzen
\renewcommand*\listfigurename{Abbildungsverzeichnis}
% Abbildungsverzeichnis erzeugen
\listoffigures{}

% Neue Seite
\newpage

% Tabellenverzeichnis hinzufügen
% \section*{Tabellenverzeichnis}
% \addcontentsline{toc}{section}{Tabellenverzeichnis}

% Abschnittspezifische Header mit Abschnittsangabe oben rechts
\thispagestyle{Tabellenverzeichnis}

\listoftables

% Neue Seite
\newpage

% Glossar und Abkürzungsverzeichnis hinzufügen
\begin{comment}
Ein kombiniertes Fachwort- und Abkürzungsverzeichnis auf Basis des Paketes "glossaries", welches Akronyme und Fachbegriffe in einem Kapitel nacheinander ausgibt.
Nur eine Überschrift wird ins Inhaltsverzeichnis aufgenommen.
Lösung gemäß https://www.overleaf.com/learn/latex/glossaries
Abkürzungen können im Text mit \acrshort{} ausgegeben werden
Fachbegriffe mit \GLs - Wichtig: Großes G!
\end{comment}

%--------------------------------------------------------------------%
% In diesem Abschnitt werden die Abkürzungen mit \newacronym definiert
%---------------------------------------------------------------------%
\newacronym{EDV}{EDV}{Elektronische Datenverarbeitung}
\newacronym{POC}{POC}{Proof Of Concept}
\newacronym{HA}{HA}{High Aviability / Hochverfügbarkeit}
\newacronym{IaaS}{IaaS}{Infrastructure as a service}
\newacronym{SaaS}{SaaS}{Software as a Service}
\newacronym{OS}{OS}{Operating Systems / Betriebssystem}
\newacronym{VMM}{VMM}{Virtual Machine Manager}
\newacronym{VM}{VM}{Virtual Machine / Virtuelle Maschine}
\newacronym{CE}{CE}{Containerengine / Container-Engine}
\newacronym{CPU}{CPU}{Central Processing Unit / Zentrale Recheneinheit (eines Computers)}
\newacronym{RAM}{RAM}{Random Access Memory / Wahlfreier Zugriffsspeicher}
\newacronym{IoT}{IoT}{Internet of Things / Internet der Dinge}
\newacronym{UFS}{UFS}{Union File System}
\newacronym{AWS}{AWS}{Amazon Web Services}
\newacronym{ITU}{ITU}{International Telecommunication Union}
\newacronym{SIP}{SIP}{Session Initiation Protocol}
\newacronym{UAC}{UAC}{User Account Control / Nutzerkontenkontrolle}
\newacronym{CA}{CA}{Certificate Authority / Zertifizierungstelle}
\newacronym{STUN}{STUN}{Session Traversal Utilities for NAT}
\newacronym{TURN}{TURN}{Traversal Using Relays around NAT}
\newacronym{NAT}{NAT}{Network Address Translation}
\newacronym{ITU-T}{ITU-T}{International Telecommunication Union}
\newacronym{VoIP}{VoIP}{Voice-over-IP}
\newacronym{BSI}{BSI}{Bundesamt für Sicherheit in der Informationstechnik}
\newacronym{VAF}{VAF}{Bundesverband Telekommunikation e.V}
\newacronym{QoS}{QoS}{Quality of Service}
\newacronym{IP}{IP}{Internet Protocol}
\newacronym{IPv4}{IPv4}{Internet Protocol V4}
\newacronym{IPv6}{IPv6}{Internet Protocol V6}
\newacronym{Payload}{Payload}{Nutzdaten}
\newacronym{UM}{UM}{Unified Communications}
\newacronym{PSTN}{PSTN}{Public Switched Telephone Network / Öffentliches Fernemeldenetz}
\newacronym{IDS}{IDS}{Intrusion Detection System}
\newacronym{PBX}{PBX}{Private Branch Exchange, englische Bezeichnung für Telefonanlage}
\newacronym{TLS}{TLS}{Transport Layer Security}
\newacronym{HTTPS}{HTTPS}{Hyper Text Transport Protocol Secure}
\newacronym{ARP}{ARP}{Address Resolution Protocol}
\newacronym{MAC}{MAC}{Media Access Control}
\newacronym{TKA}{TKA}{Telekommunikationsanlage / Telefonanlage}
\newacronym{H.323}{H.323}{VoIP-Protokollstapel gemäß Spezifikationen der \acrshort{ITU-T}}
\newacronym{SBC}{SBC}{Session Border Controller}
\newacronym{ALG}{ALG}{Application Layer Gateway}
\newacronym{VLAN}{VLAN}{Virtual Local Area Network}
\newacronym{VPN}{VPN}{Virtual Private Network}
\newacronym{SQL}{SQL}{Structured Query Language}
\newacronym{B2BUA}{B2BUA}{Back-To-Back-User-Agent, vgl. \GLs{Back-to-Back-User-Agent}}
\newacronym{GUI}{GUI}{Graphical User Interface}
\newacronym{DNS}{DNS}{Domain Name System}
\newacronym{URL}{URL}{Uniform Ressource Locator}
\newacronym{SDN}{SDN}{Software-defined Network}
\newacronym{LAN}{LAN}{Local area network}
\newacronym{WAN}{WAN}{Wide area network}
\newacronym{IEEE}{IEEE}{Institute of Electrical and Electronics Engineers}
\newacronym{API}{API}{Application Programming Interface}
\newacronym{MPLS}{MPLS}{Multiprotocol Label Switching}
\newacronym{BGP}{BGP}{Border Gateway Protocol}
\newacronym{FCS}{FCS}{Frame Check Sequence}
\newacronym{TCP}{TCP}{Transport Control Protocol}
\newacronym{UDP}{UDP}{User Datagramm Protocol}
\newacronym{TRILL}{TRILL}{Transparent Interconnection of Lots of Links}
\newacronym{DSL}{DSL}{Digital Subscriber Line}
\newacronym{NFV}{NFV}{Network Function Virtualisation}
\newacronym{LTE}{LTE}{Long Term Evolution}
\newacronym{WLAN}{WLAN}{Wireless Local Area Network}
\newacronym{XML}{XML}{Extensible Markup Language}
\newacronym{JSON}{JSON}{JavaScript Object Notation}
\newacronym{YANG}{YANG}{Yet Another Next Generation}
\newacronym{WAN}{WAN}{Wide Area Network}
\newacronym{OSS}{OSS}{Operational Support System}
\newacronym{ITIL}{ITIL}{IT Infrastructure Libary}
\newacronym{COBIT}{COBIT}{Control Objectives for Information and Related Technology}
\newacronym{KM}{KM}{Knowledge Management}
\newacronym{CRM}{CRM}{Customer Relationship Management}
\newacronym{ERP}{ERP}{Enterprise Ressource Planning}
\newacronym{DMS}{DMS}{Document Management System}
\newacronym{OLAP}{OLAP}{Online Analytical Processing}
\newacronym{DSS}{DSS}{Decision Support System}
\newacronym{WYSIWYG}{WYSIWYG}{What You See Is What You Get}
\newacronym{CTI}{CTI}{Computer Telephony Integration}
\newacronym{UC}{UC}{Unified Communications}

%--------------------------------------------------------------------------%
% In diesem Abschnitt werden die Fachbegriffe mit \newglossaryentry definiert
%--------------------------------------------------------------------------%
\newglossaryentry{OSI/ISO-Referenzmodell}
{
    name=OSI/ISO-Referenzmodell,
    description={Hierarchisches, siebenschichtiges Modell, das die logische Struktur eines Netzwerks beschreibt. Spezifiziert als X.200-Standard der \acrshort{ITU-T}}
}
\newglossaryentry{IETF}
{
    name=IETF,
    description={Internet Engineering Task Force, eine offene Gesellschaft zur Schaffung von Standards im Internet. Vgl. \url{https://ietf.org/about/}}
}
\newglossaryentry{ISDN}
{
    name=ISDN,
    description={Integrated Systems Digital Network / Dienstintegrierendes Digitales Netzwerk}
}
\newglossaryentry{RFC}
{
    name=RFC,
    description={Request for comment. Standards bzw. Standardisierungsvorschläge der IETF}
}
\newglossaryentry{ITU-T}
{
    name=ITU-T,
    description={Standardisierungsausschus der internationalen Fernmelde-Union (International Telecommunication Union, ITU)}
}
\newglossaryentry{DDoS}
{
    name=DDoS,
    description={Distributed Denial Of Service- Angriffsmethode, bei der von vielen verteilten Endpunkten ein Angriffsziel gezielt überlastet wird}
}
\newglossaryentry{MITM}
{
    name=MITM,
    description={Man in the middle. Angriffsmethode, bei der Kommuikationsverbindungen zwischen zwei Teilnehmern von Dritten abgefangen / abgehört werden}
}
\newglossaryentry{Zertifikat}
{
    name=MITM,
    description={Ein digitales Zertifikat zur \acrshort{TLS}-Verschlüsselung gemäß dem gemäß X.509 Standard der \acrshort{ITU-T}. Vgl.\url{https://www.itu.int/rec/T-REC-X.509-201910-I}}
}
\newglossaryentry{IRC}
{
    name=IRC,
    description={Internet Relay Chat, ein textbasierter Server-Client-Chat gemäß RFC1459. Vgl. \url{https://tools.ietf.org/pdf/rfc1459.pdf}}
}
\newglossaryentry{RTP}
{
    name=RTP,
    description={Real Time Protocol, ein Protokoll zur Ende-zu-Ende-Medienübertragung über paketbasierte Netze gemäß RFC 3550. Vgl. \url{https://tools.ietf.org/pdf/rfc3550.pdf}}
}
\newglossaryentry{SRTP}
{
    name=SRTP,
    description={Verschlüsselte Version von \acrshort{RTP} gemäß RFC 3711. Vgl. \url{https://tools.ietf.org/pdf/rfc3711.pdf}}
}
\newglossaryentry{Transcodierung}
{
    name=Transcodierung,
    description={Der Vorgang der Umwandlung eines Medienstroms in einen zweiten mit anderen Eigenschaften z.B. für Paketgröße, Komprimierung etc}
}
\newglossaryentry{LDAP}
{
    name=LDAP,
    description={Lightweight Directory Access Protocol. Ein Protokoll zum Zugriff auf dateibasierte Verzeichnisdienste, ursprünglich in RFC 4511 spezifiziert. Vgl. \url{https://tools.ietf.org/pdf/rfc4511.pdf}}
}
\newglossaryentry{Back-to-Back-User-Agent}
{
    name=Back-to-Back-User-Agent,
    description={Back-to-Back-User-Agent (B2BUA). Bezeichnung für eine \acrshort{SIP}-spezifische Komponente, die gleichzeitig Server- und Client ist}
}
\newglossaryentry{Spoofing}
{
    name=Spoofing,
    description={Auch Spoofen. Sinngemäß zu etwas verfälschen bzw. etwas absichtlich fälschen}
}
\newglossaryentry{RADIUS}
{
    name=RADIUS,
    description={Remote Authentication Dial-In User Service, eine Möglichkeit zur Authentifizierung von Netzwerkteilnehmern gemäß IEEE 802.1x}
}
\newglossaryentry{Registrar}
{
    name=Registrar,
    description={Komponente eines \acrshort{VoIP}-Netzes, die aktive Verbindungen mit einem Nutzerverzeichnis verknüpft und somit Lokationsdienste ermöglicht (z.B. Authentifizierung eines Teilnehmers)}
}
\newglossaryentry{URL}
{
    name=URL,
    description={Uniform Ressource Locator. Von der \gls{IETF} spezifizierter Standard zur Adressierung von Ressourcen im Internet. Vgl. \url{https://tools.ietf.org/html/rfc1738}}
}
\newglossaryentry{Load-Balancing}
{
    name=Load-Balancing,
    description={Bezeichnet die Verlagerung von Anfragen/Berechnungen auf weniger ausgelastete Komponenten}
}
\newglossaryentry{Kernel}
{
    name=Kernel,
    description={Bezeichnet den eigentlichen Kern eines Betriebssystems}
}
\newglossaryentry{Hypervisor}
{
    name=Hypervisor,
    description={Bezeichnet eine Komponente zur Überwachung und Verwaltung virtueller Maschinen}
}
\newglossaryentry{Kernelisolierung}
{
    name=Kernelisolierung,
    description={Bezeichnet ein (Sicherheits-)Konzept, bei dem der Betriebssystemkern (Kernel) von normalen Anwendungen abgeschirmt bzw. isoliert wird, um zugriffe gewöhnlicher Anwendungen direkt auf das Betriebssystem zu vermeiden}
}
\newglossaryentry{Overhead}
{
    name=Overhead,
    description={Bezeichnet ungenutze Ressourcen bzw. unnötigen Administrations- und Verwaltungsaufwand der bei wachsender Systemgröße entsteht. Ist in der Bedeutung synonym zu z.B. Wasserkopf}
}
\newglossaryentry{Benchmarks}
{
    name=Benchmarks,
    description={Bezeichnet die Ermittlung von Kennzahlen eines Systems, meist zur Performance- und Leistungsbeurteilung von Hardware}
}
\newglossaryentry{Node}
{
    name=Node,
    description={Bezeichnet im Kontext der Containerisierung immer eine vollständige Container-Installation mit eigener Container-Engine}
}
\newglossaryentry{Engine}
{
    name=Engine,
    description={Bezeichnet den Kern einer Containerisierungslösung, den eigentlichen Container-Host}
}
\newglossaryentry{Daemon}
{
    name=Daemon,
    description={Bezeichnet einen permanent laufenden und verfügbaren Systemprozess. Unter Windows als Dienst bezeichnet}
}
\newglossaryentry{Clustering}
{
    name=Clustering,
    description={Bezeichnet die logische Bündelung von Hardware- oder Softwaresystemen für Zwecke wie Leistungserhöhung oder Ausfallsicherheit}
}
\newglossaryentry{AJAX}
{
    name=AJAX,
    description={Ajax ist ein Konzept der asymmetrischen (versetzten) Dateiübertragung zwischen einem (Web-)Server und einem Browser. AJAX ermöglicht es, Bestandteile einer Seite neu zu laden, ohne das dabei gleich die gesamte angezeigte Seite neu geladen werden muss}
}
\newglossaryentry{MSI}
{
    name=MSI,
    description={Ein \acrshort{MSI}-Paket (Microsoft Installer File) ist ein Softwareinstallationspaket für Windows-Systeme, welches benutzerdefinierte Voreinstellungen enthalten kann, die beim Installationsvorgang automatisch eingerichtet werden}
}
\newglossaryentry{Subnetz}
{
    name=Subnetz,
    description={Ein Subnetz ist ein innerhalb eines IP-Bereiches getrennt adressierbarer Bereich des Internet-Protocols}
}
\newglossaryentry{ZIP}
{
    name=ZIP,
    description={Ein Dateiformat zur verlustfreien Komprimierung von Daten}
}
\newglossaryentry{Codec}
{
    name=Codec,
    description={Als Codec (Zusammengesetzt "coder" und "decoder") bezeichnet man ein Algorithmenpaar, das Daten oder Signale digital kodiert und dekodiert. Ein Codec wie z.B. g711a enthält Informationen darüber, mit welcher Abtastrate, Häufigkeit etc. ein Signal aufgezeichnet wurde}
}
\newglossaryentry{1st Level-Support}
{
    name=1st Level-Support,
    description={Erste Ebene des Anwendersupports, an welche sich ein Benutzer direkt wenden kann}
}
\newglossaryentry{CMD}
{
    name=CMD,
    description={cmd.exe, Kommandozeileninterpreter des Windows-Betriebssystems (Offiziell als Windows-Eingabeaufforderung bezeichnet)}
}
\newglossaryentry{paketvermittelt}
{
    name=paketvermittelt,
    description={In einem paketvermittelt Datennetz werden Informationen in (Teil-)Stücken übertragen. Abhängig von den Übertragungseigenschaften des Netzes werden unterschiedlich viele Pakete erstellt und gesendet, die anschließend vom Empfänger wieder zusammengesetzt werden}
}
\newglossaryentry{Orchestrar}
{
    name=Orchestrar,
    description={Steuerungskomponente zur Verwaltung einer virtuellen Umgebung. Der Orchestrar erzeugt, löscht, startet und stoppt virtuelle Maschinen, entweder manuell oder automatisiert}
}
\newglossaryentry{Bursts}
{
    name=Bursts,
    description={Kurze, sprungartige ad-hoc Datenübertragungen}
}
\newglossaryentry{DevOps}
{
    name=DevOps,
    description={Prozessansatz aus Softwareentwicklung und IT-Systemadministration. Kunstwort aus Development und (IT) Operations. Beschreibt einen Ansatz, bei dem durch gemeinsame Zusammenarbeit von Entwicklung und Administration die Geschwindigkeit der Auslieferung von Software und die Qualität der erzeugten Software verbessert werden sollen}
}
\newglossaryentry{OpenFlow}
{
    name=OpenFlow,
    description={OpenFlow ist ein Standard von der Open Network Foundation (\url{https://www.opennetworking.org/sdn-definition/}) zum Routing von Datenpaketen in einem software-defined Network}
}
\newglossaryentry{NETCONF}
{
    name=NETCONF,
    description={NETCONF (Network Configuration Protocol) ist ein zuerst mit RFC 4741 (\url{https://tools.ietf.org/html/rfc4741}) entworfenes und mit RFC 6241 (\url{https://tools.ietf.org/html/rfc6241}) erweiteres Netzwerkprotokoll zur Konfiguration und Verwaltung von Netzwerkkomponenten auf Basis einer XML-Syntax}
}


% Abkürzungen werden nur ausgegeben, wenn sie auch verwendet wurden.
% Mit title=Abkürzungsverzeichnis kann die Bezeichnung gesetzt werden
\printglossary[type=\acronymtype,title=Abkürzungsverzeichnis]

% Seitenumbruch zwischen Abkürzungen und Fachbegriffen
\newpage

% Fachbegriffe ausgeben
\printglossary[type=main,title=Fachbegriffe]


\newpage

% Formelverzeichnis hinzufügen
% \section*{Formelverzeichnis}
\addcontentsline{toc}{section}{Formelverzeichnis}

% Abschnittspezifische Header mit Abschnittsangabe oben rechts
\thispagestyle{Formelverzeichnis}

% Neue Seite
\newpage

% Symbolverzeichnis hinzufügen
% \section*{Symbolverzeichnis}
\addcontentsline{toc}{section}{Symbolverzeichnis}

% Abschnittspezifische Header mit Abschnittsangabe oben rechts
\thispagestyle{Symbolverzeichnis}

% Neue Seite
\newpage

% Sperrvermerk hinzufügen
% \section*{Sperrvermerk}
\addcontentsline{toc}{section}{Sperrvermerk}

% Abschnittspezifische Header mit Abschnittsangabe oben rechts
\thispagestyle{Sperrvermerk}

Die vorliegende Abschlussarbeit mit dem Titel … enthält unternehmensinterne Daten der Firma … Daher ist sie nur zur Vorlage bei der FOM sowie den Begutachtern der Arbeit bestimmt. Für die Öffentlichkeit und dritte Personen darf sie nicht zugänglich sein.

\vspace{2cm}
\begin{tabularx}{\textwidth}[b]{p{5cm} X p{5cm}} \cline{1-1} \cline{3-3}
(Ort, Datum)  & & Unterschrift Danny Langenbach
\end{tabularx}

% Neue Seite
\newpage

% Englischsprachiger Sperrvermerk
% \section*{Confidentiality Clause}
\addcontentsline{toc}{section}{Confidentiality Clause}

% Abschnittspezifische Header mit Abschnittsangabe oben rechts
\thispagestyle{ConfidentialityClause}

The following assignment contains confidential data relating to internal company matters at the XYZ corporation. It is only intended, therefore, for submission by the FOM to the examiners. None of its contents may be made available to the public or to any third party as a whole or in any part without the prior written consent of the author.

\vspace{2cm}
\begin{tabularx}{\textwidth}[b]{p{5cm} X p{5cm}} \cline{1-1} \cline{3-3}
(Location, date)  & & (genuine signature) Danny Langenbach
\end{tabularx}

% Neue Seite
\newpage

% Mainmatter
\clearpage

% Ab hier Kopzeile anzeigen

% Arabische Seitennummerierung
\pagenumbering{arabic}

% 1.
\section{Einleitung}
% \addcontentsline{toc}{section}{Einleitung}
\label{Einleitung}
20 Jahre nach der Veröffentlichung des agilen Manifests \cite{eckstein_20_2021} hat agile Softwareentwicklung nicht nur den Bereich der Softwareentwicklung selbst sondern auch noch viele Bereiche darüber hinaus grundlegend verändert, beeinflusst und weiterentwickelt. Zusätzlich zur Entwicklung von iterativen und agilen Modellen (allen voran das Flagschiff \gls{SCRUM} der Softwareentwicklung im Gegensatz zu den zuvor verwendeten linear-monolithischen Modellen, hat der agile Ansatz auch Bereiche wie die IT-Administration und teilweise auch Unternehmens- und Geschäftsmodelle beeinflusst.
Software und Softwareentwicklung wurde von einem (oder mehreren) monolithischen Projekt(en) zu einem fortlaufenden Prozess, dessen Entwicklung nie vollständig zu Ende ist. Dieser Ansatz hat spätestens mit der Idee von \grqq{}Windows as a Service\glqq{} \cite{jaimeo_kurzanleitung_2021} auch einer breiten Öffentlichkeit bewusst.\newline
Agile Methoden der Softwareentwicklung bedingen neue Entwicklungsstile und Arbeitsansätze. Software fortlaufend weiterzuentwickeln  macht die Fähigkeit zur permanenten Erweiterbarkeit zu einer Voraussetzung, bekannt unter den Schlagwörtern \glqq{} Continuous Development\grqq{} und \glqq{} Continuous Integration\grqq{}  (\acrshort{CD} / \acrshort{CI}). Geschäftsmodelle wie der zuvor erwähnte \glqq{}Windows as a Service\grqq{}-Ansatz oder auch generell die verschiedenen cloudbasierten Hostingansätze (Vgl. \acrshort{PaaS} und \acrshort{SaaS}) bedingen zugleich noch einen weiteren Aspekt als hinreichende Voraussetzung für das Funktionieren eines derartigen Geschäftsmodells: Kontinuierliche Verfügbarmachung - \glqq{}Continuous Delivery\grqq{} (\acrshort{CD}).\newline
Im Kontext dieser Herausforderungen wird häufig das Konzept \glqq{}\acrshort{DevOps}\grqq{} geannt und von einer \acrshort{CI}/\acrshort{CD}-Pipeline (teilw. auch \acrshort{CI}/\acrshort{CD}/\acrshort{CD}).
Doch was diese Schlagworte genau bedeuten oder wie überhaupt ihre Umsetzung in der Praxis aussieht, bleibt
häufig vage. Trotzdem hat sich auf dem Markt eine Vielzahl entsprechender Tools etabliert.
Diese Vielfalt reicht von den großen kommerziellen Anbietern wie Microsoft (Azure) und Amazon
(AWS) bis in den OpenSource-Bereich (z.B. Jenkins) und von riesigen Plattformen (z.B. RedHat Enterprise) bis hin zu Anwendungen, mit denen schon einzelne Personen ganze \acrshort{CI}/\acrshort{CD}-Pipelines administrieren können (z.B. Gitlab) und auch Personen ohne Programmierkenntnisse einen niederschwelligen Einstieg finden (z.B. NodeRed).
\subsection{Motivation}
% \addcontentsline{toc}{subsection}{Motivation}
\label{Motivation}
Allerspätestens mit dem weltweiten Wachstum cloudbasierter Anwendungen und dem \glqq{}Everything as a Service\grqq{}-Ansatz (\acrshort{XaaS}) hat agile Softwareentwicklung und damit auch die zuvor genannten Konzepte von \acrshort{DevOps} und \acrshort{CD}/\acrshort{CI} Einzug in den Mittelstand gehalten. Vor dem Hintergrund eigener \glqq{}as a Service\grqq{}-Ansätze im Unternehmen des Autors ist diese Seminararbeit entstanden. Das Unternehmen des Autors ist in der \acrshort{ITK}-Branche tätig. Besonders die Telekommunikation ist dabei ein Bereich der besonders von monolithischen Lösungen und Produkten großer Hersteller geprägt ist.  Doch diese Vormachtstellung schwindet und auch die großen Hersteller setzten auf immer flexiblere Produkte und Cloudansätze.

\subsection{Methode}
% \addcontentsline{toc}{subsection}{Methode}
\label{Methode}
Im Rahmen dieser Seminararbeit soll über eine qualitative Literaturrecherche dargestellt werden, welche genauen Werkzeuge und Methoden sich hinter den in der Einleitung genannten Schlagworten verbergen und welche Vorteile und Potenziale diese bieten.
Für einen theoretischen Rahmen werden dabei zuerst die Modelle und Ideen der klassischen Softwareentwicklung sowie der agilen Softwareentwicklung dargestellt. Auf dieser Grundlage erfolgt eine Definition der in der Einleitung genannten Schlagwort von \gls{DevOps} und \acrshort{CD}/\acrshort{CI}.
Anschließend werden sowohl die technischen als auch betriebswirtschaftlichen Potenziale der genannten Konzepte betrachtet. Darauffolgend wird die Umsetzung in der Praxis anhand von Fallbeispielen (Usecases) erläutert. Teil dieser Erläuterung ist die exemplarische Darstellung, welche konkreten Lösungen und Produkte aus dem kommerziellen oder Opensource-Bereich  auf dem Markt vorhanden sind.
Abschließend wird evaluiert welche Voraussetzungen aus der Praxis  für \acrshort{CD}/\acrshort{CI}-Pipelines definiert werden können und es erfolgt ein Fazit.
\subsection{Ziel}
% \addcontentsline{toc}{subsection}{Ziel}
\label{Ziel}

% 2.
\section{Klassische Softwareentwicklung}
% \addcontentsline{toc}{section}{Klassische Softwareentwicklung}
\label{Klassische Softwareentwicklung}
Die nachfolgenden Abschnitte erläutern die theoretischen Grundlagen klassischer (nicht-agiler) Softwarentwicklung. Dies umfasst die verwendeten Prinzipien, Methoden und Werkzeuge und eine Auswahl der markantesten Entwicklungsmodelle und eine Herleitung der Begriffe von den allgemeinen theoretischen Grundlagen zu den praktischen Ausprägungen.
\subsection{Generische Softwareentwicklunsmodelle}
% \addcontentsline{toc}{subsection}{Generische Softwareentwicklunsmodelle}
\label{Generische Softwareentwicklunsmodelle}
Grundsätzlich folgt jeder Prozess der Softwareentwicklung den gleichen grundlegenden Schritten. Diese werden in der nachstehenden Grafik in Anlehnung an \cite[Abb. 9.1]{Ludwig_Lichter_2013}: 

Jede der Phasen erfüllt dabei feste Aufgaben:
\begin{table}
	Phase & Aufgabe\\
	Analyse & Aufnahme der Anforderungen an eine Software und der (Projekt-) Umgebung. \glqq{}Konkretisierung der Analyse\grqq{} \cite[S. 155]{Ludwig_Lichter_2013}\\
	Spezifikation & Präzisierung / Verschriftlichung der Analyse\\
	Grobentwurf & Festlegung der groben Programmstruktur\\
	Feinentwurf & Präzisierung des Grobentwurfs, Festlegung der genauen Implementierung\\
	Codierung & Umsetzung des Feinentwurfs in Code, Tests des Codes (Unit-Codes)\\
	Integration, Test und Abnahme & Einpassung des Produktes in die Produktivumgebung\\
\end{table}
Diese Phasen sind in sich abgeschlossen und haben definierte Übergabepunkte (Artefakte) bei einem Wechsel zwischen zwei Phasen \cite[S. 155]{Ludwig_Lichter_2013}. Hierzu sei erwähnt, dass es sich um eine vereinfachende und idealisierte Betrachtung handelt. Der feste \textit{Top-to-Bottom}-Ansatz ist für die meisten praktischen Anwendungsfälle zu simpel und wird veränderbaren oder wechselnden Anforderungen nicht gerecht. Diese Einschränkung hat bereits  Dr. Winston W. Royce als Erfinder des Wasserfallmodells \cite{royce1987managing} erkannt und ihr Rechnung getragen. Royce selbst hat eine Wiederholung der Modelldurchläufe vorgeschlagen \cite{Larmann_Basili_2003} und damit das lineare Modell iterativ erweitert. Die nachstehende Grafik verdeutlich dies in Anlehnung an \cite[Abb. 3]{royce1987managing}:



Ein iterativer Ansatz ermöglicht es, auf Veränderungen interner und externer Faktoren in einem Softwareprojekt zu reagieren. Dabei ist es zunächst unerheblich, welche Faktoren dies sind (Externe wie z.B. veränderte Anforderungen oder interne wie z.B. Wechsel einer Programmierungsmethode), entscheidend ist nur die Phase, in welcher die Veränderungen auftreten. An diesem Punkt kann die einzelne Phase (oder alle bis dahin durchlaufenen Phasen) erneut durchlaufen werden.
Dieser lineare Ablauf wird im Spiralmodell weiter erweitert:

// Originalquelle Spiralmodell
\cite[Abb. 2]{boehm_spiral_1988} 

Von diesen Modellen können weitere Formen abgeleitet werden. Daher werden sie als generische Entwicklungsmodelle bezeichnet.

\subsection{Methode}
% \addcontentsline{toc}{subsection}{Methode}
\label{Methode}
Im Rahmen dieser Seminararbeit soll über eine qualitative Literaturrecherche dargestellt werden, welche genauen Werkzeuge und Methoden sich hinter den in der Einleitung genannten Schlagworten verbergen und welche Vorteile und Potenziale diese bieten.
Für einen theoretischen Rahmen werden dabei zuerst die Modelle und Ideen der klassischen Softwareentwicklung sowie der agilen Softwareentwicklung dargestellt. Auf dieser Grundlage erfolgt eine Definition der in der Einleitung genannten Schlagwort von \gls{DevOps} und \acrshort{CD}/\acrshort{CI}.
Anschließend werden sowohl die technischen als auch betriebswirtschaftlichen Potenziale der genannten Konzepte betrachtet. Darauffolgend wird die Umsetzung in der Praxis anhand von Fallbeispielen (Usecases) erläutert. Teil dieser Erläuterung ist die exemplarische Darstellung, welche konkreten Lösungen und Produkte aus dem kommerziellen oder Opensource-Bereich  auf dem Markt vorhanden sind.
Abschließend wird evaluiert welche Voraussetzungen aus der Praxis  für \acrshort{CD}/\acrshort{CI}-Pipelines definiert werden können und es erfolgt ein Fazit.
% 3.
\section{Agile Softwareentwicklung}
% \addcontentsline{toc}{section}{Agile Softwareentwicklung}
\label{Agile Softwareentwicklung}
Die nachfolgenden Abschnitt betrachten die historischen Hintergründe und theoretischen Grundlagen agiler Softwareentwicklung. Ebenso werden die zentralen Ideen agiler Modelle näher betrachtet und es erfolgt die Überleitung zu den Begriffen \acrshort{DevOps} und \acrshort{CI}/\acrshort{CD}.   
\subsection{Agile Grundsätze - Das agile Manifest}
% \addcontentsline{toc}{subsection}{Agile Grundsätze - Das agile Manifest}
\label{Agile Grundsätze - Das agile Manifest}
Die zuvor aufgezeigten Defizite der klassischen Softwareentwicklung führten im Jahr 2001 zur Entstehung eines neuen Entwicklungsansatzes. Unter dem Schlagwort des agilen Manifests 

\subsection{Agile Softwareentwicklungsmodelle}
% \addcontentsline{toc}{subsection}{Agile Softwareentwicklungsmodelle}
\label{Agile Softwareentwicklungsmodelle}
\subsection{DevOps und CI/CD}
% \addcontentsline{toc}{subsection}{DevOps und CI/CD}
\label{DevOps und CI/CD}
\gls{DevOps} ist der Ansatz der agilen Softwareentwicklung weitergedacht in den Betriebsalltag einer Software. \gls{DevOps} selbst ist ein Kunstwort aus \glqq{}Development\grqq{} und \glqq{}Operations\grqq{}. \gls{DevOps} bezeichnet die direkte Integration von Unternehmenskultur und unterstützender \acrshort{IT}-Prozesse in den Softwareentwicklungsprozess \cite{halstenberg_devops_2020}. Diese unterstützenden \acrshort{IT}-Prozesse umfassen oftmals Aufgaben wie Bereitstellung der Umgebung für den Test- und Wirkbetrieb einer Software, Verfügbarmachung einer Software für den Endkunden und Betreuung der Anwendung im Alltag oder auch Qualitätssicherung \cite{DevOps_Definition_AWS}. \acrshort{DevOps} soll dabei die Entwicklung und Bereitstellung von Software erleichtern und die damit verbundene Time-to-Market (Bereitstellungszeit) erheblich verkürzen. Gleichzeitig dienen diese Vorteile auch der Unterstützung anderer Phasen im Softwarelebenszyklus, insbesondere dem Patch-Management und der Wartung \cite{DevOps_Definition_Microsoft}.\newline
\gls{DevOps} ist dabei mehr als nur ein Neuordnen vorhandener Abläufe. Der Ansatz umfasst nicht nur technische Änderungen, sondern Änderungen der gesamten Unternehmenskultur: Vorher getrennte Abläufe werden miteinander vernetzt und vorher getrennte Zuständigkeiten entfallen. Abteilungen und Organisationseinheiten müssen nicht nur mehr miteinander kommunizieren, sondern müssen füreinander transparent sein. Zudem erhalten sie untereinander Einblick in Prozesse und Gebiete, für die sie eigentlich nicht zuständig sind \cite{DevOps_Definition_Microsoft} \cite{DevOps_Definition_AWS}.\newline
Vor diesem Hintergrund sind die Schlagworte \glqq{}Continuous Development\grqq{} und \glqq{}Continuous Integration\grqq{} Ausprägungen der \gls{DevOps}-Philosophie. Continuous Development (\acrshort{CD}) bezeichnet den iterativ-inkrementellen Ansatz der Softwareentwicklung, der allen agilen Modellen und damit auch dem \gls{DevOps}-Ansatz zugrunde liegt. Dabei geht es um die permanente und zyklische Weiterentwicklung der Code-Basis einer Software. Continuous Integration (\acrshort{CI}) bezeichnet die dabei stattfinde, fortlaufende Integration von Änderungen und Erweiterungen in die bestehende Anwendungen. \acrshort{CD} verfügt aber noch über eine weitere Bedeutung: Continuous Delivery. Delivery, also die Auslieferung oder Bereitstellung der Software, kann sich dabei auf die Bereitstellung des fertigen Produktes z.B. in Form eines Datenträgers oder Installationsimages beschränken. Der Ansatz kann aber auch erheblich weitergedacht werden. Dadurch sind Szenarien möglich, bei denen eine Änderung in einer direkten Kette vom ausführenden Programmierer über die Versionsverwaltung, Build-Umgebung, Test-Umgebung und Bereitstellung direkt in ein Produktivsystem im eigenen Unternehmen oder bei einem Kunden geladen wird. Diese Kette wird als \acrshort{CI}/\acrshort{CD}-Pipeline bezeichnet \cite{DevOps_Definition_Microsoft}  \cite{DevOps_Definition_AWS} \cite{atlassian_CivsCDvsCD_nodate} \cite{NodeRed_CICD_nodate}. 

% 4.
\section{Potenziale von CI/CD}
% \addcontentsline{toc}{section}{Potenziale von CI/CD}
\label{Potenziale von CI/CD}
Der nachfolgende Abschnitt betrachtet die Potenziale, die der Einsatz von \acrshort{CI}/\acrshort{CD} sowohl aus betriebswirtschaftlicher, als auch aus technischer Sicht bietet.
\subsection{Betriebswirtschaftliche Potenziale}
% \addcontentsline{toc}{subsection}{Betriebswirtschaftliche Potenziale}
\label{Betriebswirtschaftliche Potenziale}
\subsection{Technische Potenziale}
% \addcontentsline{toc}{subsection}{Technische Potenziale}
\label{Technische Potenziale}
Ebenso wie betriebswirtschaftlich, bietet eine \gls{DevOps}-Kultur auch technische Vorteile. Dies umfasst auch Erleichterungen für die technischen Mitarbeiter, wie Programmierer oder Administratoren. \gls{DevOps} hat einen stark Mensch-bezogenen Ansatz in seinen Grundsätzen des offenen und gleichberechtigen Zusammenarbeitens. Integraler Bestandteil von \gls{DevOps} ist das Auflösen von \glqq{}Silos\grqq{} \cite{leite_survey_2020}, also Bastionen begrenzten Wissens. Dieser kollaborative Ansatz bietet für Mitarbeiter einerseits die Möglichkeit, neue Erfahrungen zu sammeln und Interessen an Themen jenseits des eigenen Fachbereiches auszuleben \cite{leite_survey_2020}. Da \gls{DevOps} zeitgleich auf die Grundgedanken des agilen Manifests zurückgeht, ist die Idee regelmäßiger Arbeitszeiten und der Vermeidung exzessiver Belastung an einzelnen Stellen immanent (Vgl. \glqq{}gleichmäßiges Tempo\grqq{}, \ref{Betriebswirtschaftliche Potenziale}).
Zeitgleich entlastet \acrshort{CI}/\acrshort{CD} über den Grad der Automatisierung und die Art der Architektur die einzelnen Entwickler: Software muss so gebaut sein, dass sie in Teilen iterativ verändert werden kann, ohne Auswirkungen auf das Gesamtprojekt zu haben (\glqq{}Microservice\grqq{} \cite[S. 14]{leite_survey_2020}). Die dadurch in die Software eingebrachte Modularität macht es nicht mehr erforderlich, jede Facette eines Projektes zu kennen. Änderungen am Code werden dadurch einfacher und die Einstiegshürden für neue oder fachfremde Entwickler geringer. Dies betrifft ebenfalls den notwendigen Aufwand zur Fehlersuche und -Beseitigung (Bugfixing) \cite{hilton_usage_2016}.

//Todo
% zhao_impact_2017
% rahman_characterizing_2018


% 5.
\section{UseCases aus der Praxis}
% \addcontentsline{toc}{section}{UseCases aus der Praxis}
\label{UseCases aus der Praxis}
Die nachfolgenden Abschnitte stellen exemplarisch Tools und Anwendungfällebieten (UseCases) zu \acrshort{CI}/\acrshort{CD} aus der Praxis vor. Dabei wird eine Unterscheidung zwischen Anwendungsfällen, in denen proprietäre Lösungen und Anwendungsfällen, in denen Opensource-Lösungen zum Einsatz kommen, getroffen. Diese Unterscheidung ist der Bandbreite an möglichen Lösungen und Tools geschuldet. U.a. umfassen die proprietären Lösungen auch die (fast) vollautomatisierten Cloudplattformen von Amazon (\acrshort{AWS}) und Microsoft (Azure).
\subsection{Microsoft Azure}
% \addcontentsline{toc}{subsection}{Microsoft Azure}
\label{Microsoft Azure}

Geringere Bereitstellungskosten (wenn die CI CD Infrastruktur in eine der großen Cloudplattformen ausgelagert wird)

\subsection{Amazon Web Services}
% \addcontentsline{toc}{subsection}{Amazon Web Services}
\label{Amazon Web Services}
\subsection{OpenSource-Tools}
% \addcontentsline{toc}{subsection}{OpenSource-Tools}
\label{OpenSource-Tools}

% 6.
\section{Benötigte Funktionen für eine CI/CD-Pipeline}
% \addcontentsline{toc}{section}{Benötigte Funktionen für eine CI/CD-Pipeline}
\label{Benoetigte Funktionen für eine CI/CD-Pipeline}
Die vorangegangenen Abschnitte haben einen Eindruck vermittelt, wie langwierig der Weg von klassischer über agile Softwareentwicklung hin zu \gls{DevOps} und \acrshort{CI}/\acrshort{CD} war. Die Abschnitte \ref{Anwendungen mit proprietärer Software}  und \ref{Anwendungen mit Opensource-Software} haben einen Eindruck vermittelt, welche Vielzahl von Anwendungen es in diesem weiten Themenfeld gibt. Abschließend ist nun zu betrachten, welche Elemente zwischen den Lösungen gleich und damit integraler Bestandteil einer \acrshort{CI}/\acrshort{CD}-Pipeline sind.\newline
Eine \acrshort{CI}/\acrshort{CD}-Pipeline \glqq{}verbindet\grqq{} die folgenden Entwicklungsphasen miteinander (In Anlehnung an \cite{redhat_cicd_pipline} und \cite{meyer_continuous_2014}):
\begin{enumerate}
    \item Continuous Integration
    \begin{enumerate}
        \item Erstellen / Entwicklung
        \item Test (lokal)
        \item Zusammenführen (mit dem Rest der Code-Basis, \glqq{}Merging\grqq{})
    \end{enumerate}
    \item Continuous Delivery
        \begin{enumerate}
        \item Bereitstellung
        \item Automatische Veröffentlichung 
    \end{enumerate}
    \item Continuous Deployment
        \begin{enumerate}
        \item Bereitstellung in Zielumgebung
        \item Zielumgebung kann lokal oder entfernt (remote) im eigenen Unternehmen oder beim Kunden sein
    \end{enumerate}
\end{enumerate}
Aus diesem Ablauf ergeben sich die folgenden Elemente, die in einem derartigen Ablauf notwendig sind:
% !h um die automatische Anordnung am Seitenanfang zu unterdrücken
% https://de.overleaf.com/learn/latex/Tables#Positioning_tables
% \begin{table]} ... \end{table} damit diese Tabelle im Tabellenverzeichnis aufgenommen wird
% Landscape für Querausrichtung der Tabelle
    \begin{table}[!h]
        \centering
        \begin{tabular}{|p{3cm}|p{4cm}|p{7cm}|}
            \hline
            Abschnitt & Phase & Tool\\
            \hline
            Continuous Integration & Entwicklung & Lokale \acrshort{IDE}, (verteiltes) Versionskontrollsystem z.B. Git\\
            \hline
            Continuous Integration & Test & Lokaler Compiler / Interpreter der jeweiligen Sprache, lokale Tests\\
            \hline
            Continuous Integration & Zusammenführen (Merge) & (verteiltes) Versionskontrollsystem z.B. Git\\
            \hline
            Continuous Delivery & Release & Buildserver z.B. Jenkins\\
            \hline
            Continuous Delivery & Bereitstellung & Repository / Versionsverwaltung für die fertige Software, z.B. Nexus \cite{zanini_integrating_2018}\\
            \hline
            Continuous Deployment & Rollout in Test- oder Produktivumgebung & Softwareorchestrierungstool\\
            \hline
        \end{tabular}
            \caption{Phasen und Elemente einer \acrshort{CI}/\acrshort{CD}-Pipeline}
            \label{Tabelle:Phasen und Elemente einer CICD-Pipeline}
    \end{table}
\newpage
Diese relativ kurze Liste der notwendigen Voraussetzungen ist einerseits bewusst vereinfachend gehalten, zeigt aber andererseits auch, mit wie wenigen Tools eine \acrshort{CI}/\acrshort{CD}-Pipeline etabliert werden kann. Aufgrund der vereinfachenden Darstellung ist auch davon auszugehen, das reale Implementierungen von \acrshort{CI}/\acrshort{CD}-Pipelines entsprechend komplexer sind.

7.
\section{Fazit}
% \addcontentsline{toc}{section}{Fazit}
\label{Fazit}
Das Thema \gls{DevOps} ist aktuell. Die Anzahl wissenschaftlicher Publikationen über \gls{DevOps} steigt seit einigen Jahren permanent \cite{leite_survey_2020} und durch die Präsenz von Lösungen mit niedrigen Einstiegshürden (z.B. Nodered \cite{nodered_about}, Github \cite{DevOps_Definition_Microsoft}/Gitlab \cite{gitlab_devops}) werden die Arbeits- und Funktionsweisen für einen breiteren Personenkreis zugängig. Dabei muss es sich nicht mehr nur um Programmierer handeln, wie das Beispiel von NodeRed zeigt.
\gls{DevOps} ist außerdem Wettberwerbs- und Wirtschftsfaktor: Entwicklung mit \acrshort{CI}-Techniken führen zu schnelleren Release-Zyklen und höheren Release-Zahlen \cite{forsgren_devops_2015} \cite{hilton_usage_2016}. Damit ermöglicht \gls{DevOps} einerseits eine höhere Innovationsdichte und schnellere Reaktionen auf neue Trends (Vgl. Verkürzung der \gls{Time-to-Market} in Abs. \ref{Betriebswirtschaftliche Potenziale}), andererseits aber auch geringere Reaktionszeiten im Fall von Fehlern oder Problemen \cite{zhao_impact_2017}. \gls{DevOps} und die dadurch im Entwicklungsprozess eingebrachte Transparenz ist damit Marketingvorteil (\acrshort{USP}).
\gls{DevOps} ist außerdem integrativ und kollaborativ, sowie interdisziplinär. Über den gesamten Weg einer \acrshort{CI}/\acrshort{CD}-Pipeline berührt das Thema unterschiedlichste Unternehmensbereiche von Management über Entwicklung und Administration bis zum Personalwesen. Gleichzeitig umfasst das Thema damit Gebiete von Management- und Mitarbeiterqualifikation und notwendigen Softskills bis hin zu Unternehmensstrukturen, Ablauf- und Aufbauorganisation. Zu diesen Managementherausforderungen kommen dann die technischen Aspekte, wie der Aufbau von Entwicklungs- und Testumgebungen, Releasezyklen und -Managment und die Administration der \acrshort{CI}/\acrshort{CD}-Pipeline im Wirkbetrieb \cite[Abb. 5 bis 8]{leite_survey_2020}. 
Mit diesen umfangreichen Implikationen in die unterschiedlichsten Aspekte eines Unternehmens ist  \gls{DevOps} vielmehr eine Unternehmenskultur statt eine Ansammlung einzelner Prozesse und Praktiken \cite{DevOps_Definition_Microsoft} \cite{DevOps_Definition_AWS}. Unternehmenskultur impliziert damit auch, dass diese Philosphie und alle damit verbundenen Bereiche in einem Unternehmen, von Management und Führungsebene angefangen, gelebt werden müssen. Damit geht es für das Management vor allem darum eine Umgebung zu schaffen, in der die \gls{DevOps}-Prinzipien von Kollaboration, Transparenz, Fehlertoleranz bis zu interdisziplinärem Denken gelebt werden können \cite[Abschnitt 7.2]{leite_survey_2020}.
Aus diesem Punkt ergeben sich dann auch die Herausforderungen bei der Etablierung einer \acrshort{DevOps}-Kultur. Da eine Kultur etwas Immaterielles darstellt, gibt es keine Patentlösungen, wie derartige Umgebungen in der Praxis umgesetzt werden sollen.
Entsprechend ist der Weg bis zur vollständigen Etablierung von \gls{DevOps} geprägt von \glqq{}trial and error\grqq{}. Da es nicht \textit{die Musterlösung} gibt, muss jedes Unternehmen selbst herausfinden, welche Strategien, Methoden und Werkzeuge bei der Etablierung von \gls{DevOps} geeignet sind. Auf Grundlage der zuvor genannten Quellen kann jedoch festgehalten werden, dass allein die Verwendung einer Versionsverwaltung noch kein \acrshort{CI} darstellt, sondern dieser Ansatz wesentlich weiter gefasst ist. Dies birgt natürlich auch das Risiko des Scheiterns und des damit einhergehenden potenziellen Verlusts bereits getätigter Investitionen, entweder in Personal oder Infrastruktur. Zusätzlich lässt sich argumentieren, dass \gls{DevOps} und \acrshort{CI}/\acrshort{CD} nicht für den Einsatz in jedem Projekt und jeder Umgebung geeignet ist, doch die dieser Arbeit zu Grunde liegenden Quellen bestätigen dies nicht \cite[Abb. 2 und 3]{hilton_usage_2016}. Zusätzlich können Investitionsrisiken minimiert, bzw. ausgelagert werden, sofern auf fertige Plattformlösungen Dritter (Vgl. \ref{Anwendungen mit proprietärer Software}) zurückgegriffen wird.\newline
Zusammenfassend kann abschließend festgehalten werden, dass in der \gls{DevOps}-Philosophie und dem zugrunde liegenden agilen Manifest \cite{beck_manifest_2001} große Potenziale für Softwareentwicklung in jedem Maßstab liegen. Durch die hohe Aktualität des Themas sind Lösungen mit niedrigen Einstiegskosten weit verbreitet und einfach zugängig. Damit lassen sich Investitionsrisiken im Vergleich zu den Potenzialen minimieren. Dennoch ist nicht ausgeschlossen, dass die Etablierung einer \acrshort{DevOps}-Kultur in einem Unternehmen auch scheitern kann. Dies ist ein Aspekt, der im Umfang dieser Arbeit nicht tiefergehend bearbeitet wurde.
Die Frage, wie \gls{DevOps} daher in Unternehmen eingeführt werden kann und welche Schritte dafür als technisch-organisatorische Maßnahmen zur Risikovermeidung nötig sind, ist damit eine relevante Frage für weitere Forschung. \gls{DevOps} insgesamt bietet ein gewaltiges Feld für weitere Forschung sowohl qualitativer (Fallstudien, Experimente) und quantitativer Art (Literaturanlyse) und mit der weitergehenden Verbreitung dieser Prinzipien ist nicht absehbar, das dieser Trend alsbald endet.


% Anhang / Backmatter
\clearpage

% Römische Seitennummerierung
\pagenumbering{roman}
\setcounter{page}{9}

% Literaturverzeichnis
% Stil des Literaturverzeichnisses
% IEEEtran
\bibliographystyle{IEEEtran}
\bibliography{Inhalt/03Anhang/Literaturverzeichnis}

\clearpage

% Rechtsprechungsverzeichnis einfügen
% \include{Inhalt/03Anhang/Rechtsprechungsverzeichnis}

% Quellenverzeichnis einfügen
% \include{Inhalt/03Anhang/Quellenverzeichnis}

% Deutschsprachige ehrenwörtliche Erklärung einfügen
\include{Inhalt/03Anhang/EhrenwoertlicheErklaerungGER}

% Englischsprachhige ehrenwörtliche Erklärung einfügen
% \section*{Declaration in lieu of oath}
\addcontentsline{toc}{section}{Declaration in lieu of oath}
I hereby declare that I produced the submitted paper with no assistance from any other party and without the use of any unauthorized aids and, in particular, that I have marked as quotations all passages, which are reproduced verbatim or nearby-verbatim from publications. Also, I declare that the submitted print version of this thesis is identical with its digital version. Further, I declare that this thesis has never been submitted before to any examination board in either its present form or in any other sim-ilar version. I herewith agree\sout{/disagree} that this thesis may be published. I herewith consent that this thesis may be uploaded to the server of external contractors for the purpose of submitting it to the contractors’ plagiarism detection systems. Uploading this thesis for the purpose of submitting it to plagiarism detection systems is not a form of publication.

\vspace{2cm}
\begin{tabularx}{\textwidth}[b]{p{5cm} X p{5cm}} \cline{1-1} \cline{3-3}
(Location, date)  & & (genuine signature) Danny Langenbach
\end{tabularx}

\end{document}